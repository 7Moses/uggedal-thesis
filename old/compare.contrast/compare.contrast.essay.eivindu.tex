\documentclass[12pt,a4paper]{article}
\usepackage[T1]{fontenc}
\setcounter{secnumdepth}{-1}

\title{Pointer Sharing and Interaction Trails
       as Methods for Social Navigation}
\author{Eivind Uggedal\\
        \texttt{eivindu@ifi.uio.no}}
\date{}

\begin{document}

\maketitle{}

The aim of this essay is to compare and contrast two papers in the field of
social navigation in the context of the Web. First there will be summaries
of the papers, highlighting what I think are the most important aspects.
Then I'll compare the complementary aspects and contrast the contradictory
features of these two articles. Lastly I'll discuss my personal
opinions of the subject and how it all relates to the state of the modern web
infamously known as ``Web 2.0''.

\section{Summary of Articles}

I decided to review two of the most cited papers in this field. While both of
them are nearly a decade old they both have important insights which could be
just as valuable today as when they were written.

\subsection{Supporting Social Navigation on the World Wide Web}
The first article \cite{dieberger97} describes navigational behavior on the
web, the state of social navigation support in the tools of that time, and
reviews two prototype systems with social navigation features.

Dieberger is one of the first to compare social navigation in our concrete
real world to activities taking place on the Internet and particularly on the
World Wide Web. As he points out the term social navigation was coined as the
act when:
``\ldots movement from one item to another is provoked as an artifact of the
activity of another or a group of others.'' \cite{dourish94} Dieberger builds
on this definition and looks at the problem from a more practical standpoint.

\subsubsection{Navigational Behavior on the Web}

The author describes examples of social navigation observable on the Web where
pointers characterized as URLs are exchanged and highlights the superiority of
such forms of navigation compared to aimlessly browsing the Web. This sharing
of pointers to resources can be done deliberately by sending such
items trough interaction systems as e-mail both where the the receiver
actively requested them or just passing them around hoping that they
will be useful for the recipient. One can also take a more passive approach
where pointer pages are published on the Web for everyone to benefit from in
their search for relevant information.

Users need a clear notion of structure for the available information due to
the Internets size and increasing growth. Dieberger sees pointer collections
shared by both individuals and corporations as a solution to this
problem combined with readily available meta information of the Web's
different kinds of resources. This purely social process, as the author
describes it, is important since it combats the tedious process of searching for
information and are uncolored of commercial interests.

\subsubsection{Support for Interaction on the Internet}

The majority of e-mail and newsgroup systems of the time had little support
for distinguishing URLs from other textual content making it hard for users to
easily share and use these elements. Chat systems offered sharing of URLs in
real time but also these systems offered unnatural handling of them.
Apart from verbosely talking about the resources there was not any real
interaction present. One could not for instance point out certain parts of a
document explains Dieberger.

There is also problems with usage of URLs in our daily world. Their rather
cryptic nature and their length makes it hard to remember them without noting
them down. Advances are being made in browsers so that the protocol (HTTP)
and sometimes the sub-domain (typically \texttt{www}) and top level domain can
be omitted from the URL when typing them in.

Dieberger makes the case for hiding URLs from the end user based on the
inherent problems with their format and how unnatural tools of the that time
supported them. Users have a need to point out entities of information in an
information system and this interaction can be made simpler and more efficient
if they can ignore the address of the resource but share and access it trough
a handle, a metaphor wrapped around the complex URL as the author calls it.
This handle would also need to provide easily accessible meta information
about the resource it represents so that users can make an informed decision
about whether the document is worth visiting without actually visiting and
reviewing the document itself. Lastly this page handle have to be an
integrated part of the information system it's implemented in so that it
can function as unobtrusive as possible.

Awareness of other users on the Web can create opportunities for several forms
of social interaction forms and thereby forms of social navigation. Dieberger
says that this perception of other individuals acts as a special type of meta
information giving feedback on where people are going, in what amounts and
who they are. It's important to be aware of the privacy issues that arises
when leveraging such information directly. By using hints about other users'
behavior indirectly one can follow a safer path which eliminate such
concerns. Nevertheless the communication systems of this time presents the Web
as a vast information repository where you're the only user, not visualizing a
shared perspective where individuals can take advantage of each other's
actions.

\subsubsection{Examples of Social Navigation Systems}

The author uses two prototypes of information systems to highlight what he
sees at the defining aspects of social navigation on the Web. First we have
the ``Juggler'' system, a text based real-time communication system
with an integrated web proxy doubling as an alternative communication
interface and historical repository. It's most evident features with
regard to social navigation support include several techniques for handling
exchange of web pointers as seamless as possible and methods of visualizing
historical aspects of information.

With the extensions made to this system it was possible for users to share a
web page they were currently viewing with the push of a single button. The
receiver(s) would then be presented with a new browser window of that exact
page automatically. The author concludes that this simple way for users to
point out web pages increased the social navigation activity.
This supports his postulation of users unwillingness to have URLs made visible
for them.

The system incorporate the idea of ``hit-counters'' on web pages
to make visible the frequency of usage for it's various navigation paths. This
turns the system into a \emph{history-enriched} \cite{hill94}
environment where so called
\emph{read wear} \cite{hill92} can be seen which points users in the direction
of the most visited and used information entities. The textual communication
system uses verbal descriptions to implement read wear, denoting both how often
read-only sources are visited and also the frequency of postings for editable
information. The author emphasizes the importance of decaying these
representations making recent information more important than older
information. The notion of read wear were taken even further in the Web
gateway visualizing frequency of usage trough color coding objects.

An alternative system to the conventional browser bookmark list
named ``Vortex'' is also used to highlight topics of social navigation. The
prototypical implementation handles URLs and associated meta-data (usage
statistics, keywords, summary etc.) as a coherent object. These objects,
referred to as handles by the author, can also be annotated so that it's
perceivable for the author which handles are the most important trough the
size of it's associated icon. Even though this system was not intentionally
made for communication and collaboration purposes such features have been
added. These come in the form of automatic functions for copying whole
pointer objects into emails with their meta-information and parsing of
URLs in incoming email into handles. Dieberger also envisions a port of
the system to a more system independent applet platform enabling users to
export their handle collections to a standard markup format, upload it to
a central server, enabling other users to contribute to this
shared collection.

\subsection{Footprints: History-Rich Tools for Information Foraging}

The second article \cite{wexelblat99} communicates the authors proposed theory
of interaction history, it's application in real world tools, and the results
of controlled usage studies. 

Wexelblat et al. contrasts the digital world of computers against our
non-digital everyday world with respect to the formers lack of history. In our
traditional world we exploit such historical information traces:
``\ldots to guide our actions, to make choices, and to find things of
importance or interest.'' \cite{wexelblat99}
Objects that contain information that make such interactions possible are
described as \emph{history-rich} according to the authors. It's important to
note that usage of interaction history in the physical world are mostly
achieved without any conscious effort and therefore feels natural to us.

This apparent lack of history in our digital systems must be sorted out such
that future users can take advantage of past users historical traces left when
they were working on solving problems similar to the current user's. This
should make problem solving easier and more efficient by recovering past
findings and avoiding mistakes made in the past. Without such systems users
have to behave like they are the first and only people using the information
-- everyone have to start from scratch in their problem solving tasks.

\subsubsection{Theoretical Interaction History Framework}

Wexelblat et al. have taken earlier insights into history-rich objects and
their usage \cite{hill92, hill94} and extended it into a theoretical framework
consisting of six properties, all characterizing interaction history based
systems:

\begin{description}
  \item[Proxemic versus distemic] describe how close people and places are
    related. A \emph{proxemic} space is transparent to it's users since it's
    signs and structures are easily comprehended. \emph{Distemic} places are on
    the other hand opaque to it's users. Translating this to interaction
    history systems means that a proxemic system relates well to it's users
    and takes advantage of past interaction experiences and knowledge.
  \item[Active versus passive] characterize the way interaction history is
    recorded and made available. One can leave records actively (saving a
    bookmark) or passively (a web client's history records). The latter
    involves no
    conscious effort and is according to Wexelblat et al. therefore the desired
    kind as users problem solving processes remains uninterrupted.
  \item[Rate/form of change] explains how history builds up trough an
    \emph{accretion} process as more and more interactions take place. History
    also fades out as time goes by and more history accumulates. It's important
    to deal with these processes so that all available information can be
    summarized in a way that makes efficient usage and access to it possible.
  \item[Degree of permeation] detail to what extent the interaction history is
    part of the history-rich object itself, ranging from inseparable to
    completely separable.
    %TODO: MABY ELABORATE ON THIS A BIT
  \item[Personal versus social] means that history either can be social of
    nature (a groups shared bookmark collection) or personal (bookmark
    collection exclusive for one individual). Wexelblat et al. feels that
    the social kind of interaction history is more valuable since most problem
    solving tasks are collaborative in nature. The authors is stressing that
    the main advantage of interaction history lies in the way it enables
    newcomers to benefit from work completed in the past.
    
  \item[Kind of information] refers to the idea that there exists infinite
    kinds of information history. The task that the users are trying to
    achieve determines what kinds of information that is of importance. The
    authors categorizes information into categories based on different uses of
    interaction history:
  \begin{description}
    \item[What] was done. Helpful if users are trying to find useful
      information, need reassurance or need guidance.
    \item[Who] has done something. Crucial for sociability, authority, and
      authenticity. 
    \item[Why] something was done. Important for knowing whether the past users
      had similar intentions, for explanation, and for learning.
    \item[How] something was done. Essential for achieving naturalness and
      transparency.
  \end{description}
\end{description}

\subsubsection{Interaction History on the Web}

The ``Footprints'' system is described by the authors as an attempt to find
the valuable aspects of interaction history in our real world and how one can
capture history in digital systems for such applications. They wanted to
validate their theoretical framework and did so by applying this theory to the
problem of how to navigate a complex information space as the Web. The system
is an active navigational aid both visualizing interaction history
information and also enables users to navigate this information. If the user
knows what he want to find Footprints can help them find their way and
explain what they've found.

The system consists of several tools integrated into a normal web browser,
using navigation transitions as their fundamental information for historical
preserving. All tools are based on a navigational metaphor:

\begin{description}
  \item[Maps] are used to show traffic trough the currently visited
    web site. Transitions from this page to other documents are visualized as
    links between several nodes. Read wear is existent trough visualization of
    a documents popularity as shades of different colors.
  \item[Trails] are ``\ldots coherent sequences of nodes followed by an
    individual.''
    \\ \cite{wexelblat99} Trails are visualized trough a path
    interface which acts as a low level view compared to the high level nature
    of the map interface. These trails are automatically recorded and the
    system uses several properties to distinguish between different paths.
    Trails are also visualized by their degree of usage, the ticker the links
    between nodes the higher the actual usage of the trail.
  \item[Signposts] are a way for users to comment on the interaction history
    they are using, giving feedback to both individual pages and trails.
\end{description}

Lastly users are given information about the usage of in-line links of web
pages trough the use of textual annotations showing the percentage of users
who followed each particular link.

\subsubsection{Usage of Interaction History Tools}

The Footprints application was tested in a controlled environment by
Wexelblat et al. where several users were given the task of locating a used
car within a given price range. One group used normal web clients while
another group used the Footprints application.
Their pre-test hypotheses were that this tool
would increase the number of a alternative car dealerships found and reduce
the number of pages visited while browsing for these. The authors also had
ambitions of the tools ability to ease the task of locating
information and understanding discovered information in addition to
increasing the level of satisfaction among users.

The experiment only partially supported the authors claims. No notable
differences between number of alternatives between the two groups were found.
But the group using the history-enriched system reported significantly lower
values of their mean page count for completing their task.

With regards to subjective responses the authors found that experienced users
(users who knew the problem domain to some extent) could to a larger degree
take advantage of interaction history in their browsing. Novice users on the
other hand was actually less satisfied with using interaction models than
normal browsing methods. Wexelblat et al. explains that experienced users
could make much better use of Footprint's representations since they had a
clear mental model of car information and thus making the application's
interaction history more proxemic.

During this experiment the authors found three different patterns of use for
Footprints. Some wandered down paths not used before within the system and
therefore received little help from the system. Others started out using
information within Footprints and took off in other directions when their
taste in cars started to differ from past users. Lastly it was observed that
some users started out without using information from Footprints, but when
their browsing brought them into trails of past users they seemed to take
advantage of their earlier work.

\section{Comparison of Articles}

Both articles describes how normal browsing behavior on the Web can be
improved by means of social navigation exemplified in several tools. They take
very different approaches to include sociability in the Web. Dieberger focuses
on improving patterns of social navigation that have widespread use while
Wexelblat et al. have created tools which support navigating the Web in
entirely new ways.

Dieberger describes tools where users actively shares information trough a
social process and then actively decides whether to use this information.
Wexelblat et al. on the other hand emphasizes on passively collecting
information that can be used actively by other users at a later stage. The
distinction between these two concepts can be exemplified as the difference
between asking or being told what direction to take in a myriad of forest
trails and following the trail that is most worn out. In the first case the
sender have to explicitly tell the receiver where to go, requiring conscious
effort. This is contrasted in the last example where several users of the
forest leave traces of their action unconsciously and without no additional
effort.

The behavior of people's structuring of the Web trough making structured
pointer pages available as described by Dieberger also applies to the
Footprints system. When people passively creates surfing trails and then
actively comment on both trails and individual documents they too are opposing
structures on the Web.

In the Juggler system people are aware of each other and can to a certain
degree know their identity as opposed to the Footprints system where all
users' behavior are merged and therefore making this an anonymous system. The
advantage of Dieberger's approach is that you by using the system can learn
which users to trust, using information provided by reputable users
before unknown people. Since the Footprints system tracks every movement of
it's users such an approach would bring forth a host of privacy issues.
Wexelblat et al. states that this is a deliberate trade-off, they wanted to
focus on interaction history and not methods for personal privacy.

As discussed earlier Dieberger proved his thesis that users do not want to see
URLs while he was working on the Juggler system (extended into his work on the
Vortex system). This same idea is implemented
throughout the Footprints system of Wexelblat et al.
Trough not explicitly stated in the article this is evident in the design
of the system. Individual documents are denoted by their page title in all of
Footprint's tools and the URL is only part of the browser's location bar.

Usage of read- and edit-wear are apparent in both the Juggler and the
Footprints application. Both use colorized visualizations and textual
representations where either is more appropriate. There seem to be differences
in what colors are used for displaying read- and edit wear. It could be
confusing for users when such presentation methods differ amongst different
applications. It would therefore be advisable to either try to establish
standard patterns or take advantage of conventions found in the wild, both on
the Web and in our real world. Developing broad conventions is difficult as
differences among cultures also have to been taken into consideration.

\section{Discussion}

This section will try to highlight the state of social navigation support in
todays applications and the Web in areas of pointer sharing and interaction
history.

\subsection{Pointer Sharing}

\subsubsection{Problems with URLs}

Dieberger describes the various problems with URL handling in the systems of
that time and that this affects how likely people are to share pointers. Today
most email clients makes URLs clickable in contrast with similar systems a
decade ago. I surveyed some of the most popular web-based clients and also
clients for Microsoft Windows, Apple OS X and Linux. The results can be seen
in Table~\ref{table:email} (p.~\pageref{table:email}).

\begin{table}[h!b!p!]
  \caption{URL Recognition Support in Email Clients}
  \label{table:email}
  \begin{center}
    \begin{small}
      \begin{tabular}{l|ccc}
        &
        \texttt{domain.tld} &
        \texttt{www.domain.tld} &
        \texttt{http://domain.tld} \\
        \hline
        Google Gmail (2007-06) & $\surd$ & $\surd$ & $\surd$ \\
        Yahoo Mail (2007-06)   &         & $\surd$ & $\surd$ \\
        MS Hotmail (2007-06)   &         &         & $\surd$ \\
        SquirrelMail (1.4.10a) &         &         & $\surd$ \\
        Moz Thunderbird (1.0)  &         & $\surd$ & $\surd$ \\
        Evolution (2.0.2)      &         & $\surd$ & $\surd$ \\
        KMail (1.7.1)          &         & $\surd$ & $\surd$ \\
        Apple Mail (2.1.1)     &         &         & $\surd$ \\
        MS Outlook (2007)      &         & $\surd$ & $\surd$ \\
      \end{tabular}
    \end{small}
  \end{center}
\end{table}

Note that this is when URLs are in plain text and not marked up as
HTML anchors. As one can see from the list all clients supported links with
protocol (HTTP) prefixes. Two thirds of the clients supported links without
protocol prefixes but with the \texttt{www} prefix. The Google Gmail web-based
client stood out from the crowd with support for all kinds of valid URLs
without protocol prefixes. Compared to the days when Dieberger wrote his
article the state of automatic URL recognition is vastly improved even
though some email clients could better serve their users by
mimicking features of state of the art clients as Google Gmail. 

Real-time communication systems as Juggler mainly comes in the form of so
called \emph{instant messaging} clients today. There exist various kinds of
instant messaging protocols with stand-alone clients that support one or more
of these protocols. Lately quite a few web-based implementations
also have appeared. It seems like their status when it comes to URL
recognition is on par with email clients even though I have not completed as
throughout a study of instant messaging clients.

% Cultural change have made URLs more commonplace. Can omit http now and
% people understand that lowercase words with a tld prefix are URL addresses to
% a large degree. Also increased emphasis on <nice URLs> have made usage of
% them directly more efficient and easy. To some degree, but not totally, this
% invalidates some of Dieberger's reasoning for making URLs hidden.

\subsubsection{Sending Pointers}

URL recognition in these communication systems used on the Web is only half
the equation. What about the individual that sends these pointers that are
recognized on arrival? Are there support for making their process more natural
in the tools of this day?

If your communication channel of choice is email you get some assistance. All
major modern web browsers (with the exception of MS Internet Explorer)
support right clicking either on the current page or
a link within the page and then selecting some sort of function like ``Send
link\ldots'' from the context menu.

Depending on your operating system, web browser, and communication
client of choice dragging and dropping pointers directly from the browser
should be possible. I do not have the complete picture of how well this is
implemented for combinations of these systems, but I can verify that such
behavior is possible on Microsoft Windows 2003, Mozilla Firefox 1.5 and
Windows Live Messenger in addition to Linux, Mozilla Firefox 2.0 and Pidgin
2.0.1.

\subsubsection{Storing Pointers}

Pointer pages as Dieberger discusses have become less popular as new services
for sharing and storing web pointers have emerged. The most popular and one of
the most influential services in this area is in my opinion
Delicious\footnote{http://del.icio.us}. This service enables one to store
bookmarks in a one click operation (in addition to an optional annotation
phase) trough browser extensions or a \emph{bookmarklet}\footnote{a small
JavaScript application stored within the URL of a browser bookmark}.
Users usually annotate such bookmarks with \emph{tags} -- single worded
keywords or labels of the document's content and purpose.

Opposed to Dieberger's notion of neatly structured pointer pages
bookmarking services as Delicious imposes user generated taxonomy on various
documents of the Web. The accuracy of such meta-data when several people
annotate the same resource has been proven to be quite good \cite{golder05}.

By harnessing these efforts completed by a large number of individuals one
can locate information both by browsing all
bookmarks of a particular user or all users and bookmarks of a certain tag or
combination of tags for either a particular user or all users.
Dieberger's proposed improvements to the Vortex system for enabling
collaborative bookmarking is now a reality with the rise of social bookmarking
sites as Delicious.

% Taking notes not supported at that time. See what various post-it like
% extensions to the Web, both server side and client side trough extensions, are
% toing today and what it means.

% Lack of meta-information on the Web. We have RDF, but no widespread use.
% Microformats as a pragmatic grassroots kind of thing are taking off in some
% communities.


\subsection{Interaction Trails}

Wexelblat et al. describes how past users' surfing trails can help new users
in their web browsing tasks. ``Trailfire''\footnote{http://trailfire.com} is a
web service that lets users create such trails with a browser extension or
bookmarklet. Surfing these trails requires nothing more than a modern
web-browser. To start a new trail you leave a mark on an interesting web
document by using a button provided by your browser extension mechanism of
choice. Marks are annotated with a title (defaults to the documents page
title) and an optional comment. The next time you visit a related page you can
choose to leave a mark on it too. You are presented with a choice of which
trail to connect this marked document to. By leaving several such marks you
create a sequence of pages that constitutes the trail.

If we compare Trailfire to Footprints we see several differences. The first
and most notable difference is Trailfire's reliance on users to actively
create trails. This has several implications. Leaving trails will be done at a
severely lower frequency since this approach introduces a conceptual overhead
for the user. On the other hand, making the process conscious for users should
result in higher quality trails, including only the truly essential documents.

Secondly the trails created in Trailfire are individualistic if browsed
without the
available browser extension. In such a usage scenario the entry points for
trails would be the trail's URL. If the users is equipped with the extension
they will be able to follow a trail if they stumble upon a web page which is
part of that trail. One would also be able to follow down the paths of other
trails when browsing a trail where part of it's pages are part of other trails
as well. With the browser extension enabled Trailfire functions like
Footprints where trails can spread into branches.

Thirdly one is aware of other users in Trailfire and can see which user
performed what action. Privacy issues which Wexelblat et al. describes in
relation to Footprints is not as relevant for Trailfire because of it's active
nature. Users chooses which documents to annotate themselves opposed to
Footprints' behavior of collecting information about all pages a user
might come across even if accessing it was unintentional.

Lastly Trailfire lacks proper visualization of various trails and their
relationships as one can find in Footprints. The navigation system introduces
a more or less linear navigational metaphor. Jumping to the previous, next or
a specific marking is your only options.

Trailfire has an advantage in the way it's neatly integrated in the browser.
The task of adding a bookmarklet requires substantially less effort than
installing a system as Footprints. This should result in more people using the
system and the amount of useful trails therefore would increase.

Wexelblat et al. talked about a proposed addition to Footprints where users
should be able to describe their purpose for the task they were carrying out.
This way the passively recorded trails had a purpose associated with them
enabling future users to better identify which trail would be most
advantageous to follow. Trailfire have support for this to some degree trough
each trails title and description.

Trailfire is an evolution of social bookmarking sites as Delicious and not
inspired by interaction history tools as Footprints. This is why we see so
large differences in their respective designs. But there are similarities --
both tools' central information structures are trails. Wexelblat et al. and
Trailfire's founder \cite{ohalloran07} say that they build upon the ideas of
trails trough digital information from the MEMEX \cite{bush45}.

There have also been work inside the domain of interaction history building
directly upon the work of Wexelblat et al. The author of the first article
discussed in this essay wrote about his experiences with
incorporating interaction history into an early \emph{wiki} system called
``CoWeb''.
\\ \cite{dieberger00} Their intentions for doing so were to include
user awareness into
the system making it a social place. This was achieved by visualizing the
recentness of document access in addition to the already present information
about the time since a document's last edit. CoWeb is therefore a passive
interaction history system.

\section{Conclusion}

We have seen that the systems we use on the Web today to some degree addresses
the problems Dieberger describes for systems that were in use a decade ago.
URLs are recognized satisfactory by most of our communication tools and they
can be handled with ease in some operation system environments. In spite of
these improvements URLs can not be manipulated as transparently as Dieberger's
Vortex system.

One can see few improvements in efforts of hiding URLs from the end users in
todays web. I'm of the opinion that URLs are a central part of the web
browsing experience and especially useful for advanced users when they are
designed intelligently and cleanly. It also seems like normal users have grown
more accustomed to URLs. Today one could expect most modern people to
distinguish lowercased wordings with dot-notation and a TLD suffix as an URL.
Advertisers often expect such knowledge of it's target audience when they
include such shortened URLs in commercials.

Storage of pointers should in my opinion be done
centralized, collaborative, and as a social  process on the Web. Mozilla's 3.0
release of the Firefox browser will see a completely new implementation of
pointer storage called \emph{places} \cite{places07} with built-inn tagging
support. Mozilla's developers have not yet decided if pointer repositories
should be stored centrally. Such an approach would make bookmarking a social
process and would make social navigation possible when using other
individuals pointers within the browsers own bookmarking engine.

Interaction history in the form of passively recording users' page transitions
is to my knowledge unique to the Footprints system. As previously discussed
the Trailfire service comes close to Footprints but falls short in terms of
visualization and unobtrusive collection of interaction data. Whether higher
quality of interaction trails can be achieved trough active acquirement
opposed to passive accumulation remains to be seen.

\section{}


\bibliographystyle{apalike}
\bibliography{citations}

\end{document} 
