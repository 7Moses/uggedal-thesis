\documentclass[12pt,a4paper]{article}

\usepackage[T1]{fontenc}

\setcounter{secnumdepth}{-1}

\author{Eivind Uggedal\\
  \texttt{eivindu@ifi.uio.no}}

\title{INF5550 -- Master Essay}

\begin{document}

\maketitle{}

The aim of this essay is to compare and contrast two papers in the field of
social navigation in the context of the web. First there will be summaries
of the papers, highligting what I think are the most important aspects.
Then I'll compare the complementary aspects and contrast the contradictory
features of these two articles. Lastly I'll discuss my personal
oppinions of the subject and how it all relates to the state of the modern web
infamously known as ``Web 2.0'' and finally come to a conclusion.

\section{Summary of Articles}

I decided to review two of the most cited papers in this field. While both of
them are nearly a decade old they both have important insights which could be
just as valid today as when they were written.

\subsection{Supporting Social Navigation on the World Wide Web}
The first article \cite{dieberger97} describes navigational behaviour on the
web, the state of social navigation support in the tools of that time and
reviews two prototype systems with social navigation features.

Dieberger is one of the first to compare social navigation in our concrete
real world to activities taking part on the Internet and particularly on the
World Wide Web. As he points out the term social navigation was coined as the
act when:
``\ldots movement from one item to another is provoked as an artefact of the
activity of another or a group of others.'' \cite{dourish94} Dieberger builds
on this definition and looks at the problem from a more practical standpoint.

\subsubsection{Navigational Behaviour on the Web}

The author describes examples of social navigation observable on the web where
pointers characterized as URLs are exchanged and highlighs the superiority of
such forms of navigation compared to aimlessly browsing the web. This sharing
of pointers to resources can be done deliberately by sending such
items trough interaction systems as e-mail both where the the receiver
actively requested them or just passing them arround hoping that they
will be usefull for the recepiant. One can also take a more passive approach
where pointer pages are published on the web for everyone to benefit from in
their search for relevant information.

Users need a clear notion of structure for the available information due to
the Internets size and inreasing growth. Diberger sees pointer collections
shared by both individuals and corporations as Yahoo as a solution to this
problem combined with readily available meta information of the webs
different kinds of resources. This purely social process as the author
describes it is important since it combats the tedious process of serching for
information and are uncolored of commercial interests.

\subsubsection{Support for Interaction on the Internet}

The majority of e-mail and newsgroup systems of the time had little support
for distinguishing URLs from other textual content making it hard for users to
easily share and use these elements. Chat systems offered sharing of URLs in
real time but also these systems offered unnatural handeling of them.
Apart from verbosely talking about the resources there was not any real
interaction. One could not for instance point out certain parts of a document
explains Dieberger.

There is also problems with usage of URLs in our daily world. Their rather
cryptic nature and their length makes it hard to remember them without noting
them down. Advances are beeing made in browsers so that the protocol (HTTP)
and sometimes the subdomain (typically \texttt{www}) and top level domain can
be ommitted from the URL when typing them in.

Dieberger makes the case for hiding URLs from the end user based on the
inherent problems with their format and how unnatural tools of the that time
supported them. Users have a need to point out entities of information in an
information system and this interaction can be made simpler and more efficient
if they can ignore the address of the resource but share and access it trough
a handle, a methapor wrapped around the complex URL as the author calls it.
This handle would also need to provide easily accessible meta information
about the resource it represents so that users can make an informed decition
about wheter the document is worth visiting without actually visiting and
reviewing the document itself. Lastly this page handle have to be an
integrated part of the information system it's implemented in so that it
can function as unobtrusive as possible.

Awareness of other users on the web can create opportunities for several forms
of social interaction forms and thereby forms of social navigation. Dieberger
means that this perception of other individuals acts as a special type of meta
information giving feedback on where people are going, in what ammounts and
who they are. It's important to be aware of the privacy issues that arises
when leveraging such information directly. By using hints about other users'
behaviour indirectly one can follow a safer path which eliminate such
concerns. Nevertheles the communication systems of this time presents the web
as a vast information repository where you're the only user, not visualizing a
shared perspective where induviduals can take advantage of eachother's
actions.

\subsubsection{Examples of Social Navigation Systems}

The author uses two prototypes of information systems to highlight what he
sees at the defining aspects of social navigation on the web. First we have
the ``Juggler'' system, a text based real-time communication system
with an integrated web proxy doubling as an alternative communication
interface and historical repository. It's most evident feautres with
regard to social navigation support include several techniques for handeling
exchange of web pointers as seamless as possible and methods of visualizing
historical aspects of information.

With the extensions made to this system it was possible for users to share a
web page they were currently viewing with the push of a single button. The
receiver(s) would then be presented with a new browser window of that exact
page automatically. The author concludes that this simple way for users to
point out web pages increased the social navigation activity of users he
surveyed even though there were some issues with the ammount of pointers
one person could get exposed to.

\subsection{Footprints: History-Rich Tools for Information Foraging}

The second article \cite{wexelblat99} communicates

\section{Comparison}

Compare here.

\section{Contrasting}

Dieberger cites Dourish (citeme) as the originator of the term social
navigation.

\section{Discussion}

Discussion related to today here.

\section{Conclusion}

Some smart conlusion.

\section{}


\bibliographystyle{apalike}
\bibliography{citations}

\end{document} 
