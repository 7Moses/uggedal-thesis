\documentclass[12pt,a4paper]{article}

\usepackage[latin1]{inputenc}
\usepackage[T1]{fontenc}

\setcounter{secnumdepth}{-1}

\author{Eivind Uggedal\\
  \texttt{eivindu@ifi.uio.no}}

\title{INF5550 -- Master Essay}

\begin{document}

\maketitle{}

The aim of this essay is to compare and contrast two papers in the field of
social navigation in the context of the web. First there will be summaries
of the papers, highligting what I think are the most important aspects.
Then I'll compare the complementary aspects and contrast the contradictory
features of these two articles. Lastly I'll discuss my personal
oppinions of the subject and how it all relates to the state of the modern web
infamously known as ``Web 2.0'' and finally come to a conclusion.

I decided to review two of the most cited papers in this field. While both of
them are nearly a decade old they both have important insights which could be
just as valid today as when they were written.

\cite{wexelblat99}
\cite{dieberger97}

\section{Summary}

\subsection{Supporting social navigation on the World Wide Web}

Dieberger is one of the first to compare social navigation in our concrete
real world to activities taking part on the Internet and particularly on the
World Wide Web. As he points out the term social navigation was coined as:
``\ldots movement from one item to another is provoked as an artefact of the
activity of another or a group of others'' \cite{dourish94}. Dieberger builds
on this definition and looks at the problem from a more practical standpoint.
He describes the state of social navigation support in the tools of that time
and reviewing two prototype systems with social navigation features.

The author describes three examples of social navigation observable on the web
which is superior to
normal browsing when one looks for specific information. First we have the
act of deliberately sharing pointers to web pages (URLs) trough e-mail
messaging. Secondly we have the more natural example copared to our every day
lives where an individual asks another person for pointers to information one
typically belive they have an essential understanding of. A third example is
published lists of web pointers found on personal web pages where an
individual with insight into a domain shares what he finds to be the most
important resources in that particular area.

Users need a clear notion of structure for the available information due to
the Internets size and inreasing growth. Diberger sees pointer collections
shared by both individuals and corporations as Yahoo as a solution to this
problem combined with available meta information of the World Wide Webs
different resources.



\subsection{Footprints: History-Rich Tools for Information Foraging}

Highlights of article here.

\section{Comparison}

Compare here.

\section{Contrasting}

Dieberger cites Dourish (citeme) as the originator of the term social
navigation.

\section{Discussion}

Discussion related to today here.

\section{Conclusion}

Some smart conlusion.

\section{}


\bibliographystyle{apalike}
\bibliography{citations}

\end{document} 
