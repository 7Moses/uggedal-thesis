\chapter{Conclusion}
\label{chapter:conclusion}

The results of the research which we have provided in this thesis can be
categorized into three venues:

\begin{enum}
  \item We have given a strucutred overview of the field of social navigation
    as seen both in academic literature and in some noticeable social web
    sites.
  \item We have provided details of how one can implement unobtrusive
    prototypes in established spaces and the feasibility of such a technical
    approach.
  \item We have contributed knowledge of how activity streams functions as
    a social navigation technique on the \urort{} web site.
\end{enum}

\section{Lessons Learnt}

Based on these three venues we'll provide the most important lessons to take
away from our research.

\subsection{Social navigation}

As viewed in academic literature social navigation can mean different things.
We proposed a new definition of social navigation based on our belief in the
importance of peers in a social navigation system.%
\sidenote{
  The definition can be found in
  \sectionref{flickr.facebook.discussion.peers}.
}
This means that the information given from other people which guide navigation
have to come from peers within the system where navigation are conducted to be
considered social navigation. Information given by the creators or 
editors of a web site are therefore not social navigation when used for
navigational purposes.
The creators can however implement structures in their web pages where users
of the system can impose information which can be used for social navigation.
One example of this divide can be found in recommender systems. Content based
recommendations is not social navigation since the information used
in the navigational process are given by the editors of the web page.
Recommendations given by collaborative filtering is on the other hand social
navigation since the navigational information is given by peers in the system.

\subsection{Unobtrusive prototyping}

\subsection{Activity streams}

\section{Future Work}

% If we had time we would have conducted a laboratory study as well. This way
% we would not have problems with the technological seeding we've seen
% since only the most technical apt were able to complete the entire study.

% We would also have tested the prototypes over a longer time frame. Favorite
% usage numbers gave us unconclusive results. With a longer usage period there
% could be possible changes in how important favorites are.
