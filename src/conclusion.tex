\chapter{Conclusion}
\label{chapter:conclusion}

The results of the research which we have provided in this thesis can be
categorized into three venues:

\begin{enum}
  \item We have given a structured overview of the field of social navigation
    as seen both in academic literature and in some noticeable social web
    sites.
  \item We have provided details of how one can implement unobtrusive
    prototypes in established spaces and the feasibility of such a technical
    approach.
  \item We have contributed knowledge of how activity streams functions as
    a social navigation technique on the \urort{} web site.
\end{enum}

Based on these three venues we'll provide the most important lessons to take
away from our research before we discuss possible future work in these fields.

\section{Lessons Learnt}

We believe there are both some theoretical and practical lessons to take away
from our research.

\subsection{Social navigation}

Social navigation can mean different things as viewed in academic literature.
We proposed a new definition of social navigation based on our belief in the
importance of peers in a social navigation system.%
\sidenote{
  The definition can be found in
  \sectionref{flickr.facebook.discussion.peers}.
}
This means that the information given from other people which guide navigation
have to come from peers within the system where navigation are conducted to be
considered social navigation.

Information given by the creators or editors of a web site
when used for navigational purposes are therefore not social navigation.
The creators can however implement structures in their web pages where users
of the system can impose information which can be used for social navigation.
One example of this divide can be found in recommender systems. Content based
recommendations is not social navigation since the information used
in the navigational process are given by the editors of the web page.
Recommendations given by collaborative filtering is on the other hand social
navigation since the navigational information is given by peers in the system.

\subsection{Unobtrusive prototyping}

Creating unobtrusive prototypes with Greasemonkey have its advantages and
disadvantages when used in real world experiments.

Greasemonkey is best fitted for situations where one don't have access to the
established web site one are prototyping on. If one have access to the inner
workings of a web site, it would probably be more efficient and easier to
implement the prototype within the established implementation. Another benefit
of modifying the web site implementation itself is the elimination of the
Greasemonkey and user-script installation process, in addition to wider
browser support.

We found the major disadvantage of using Greasemonkey in a real world
experimental setting to be this complicated installation process and limited
browser support. We contribute this as the major factors for the high
non-accomplish rates we witnessed. Having conducted experiments in a
laboratory setting where users used pre-configured machines would have
mediated this problem.

\subsection{Activity streams}

Based on our experiment with activity streams on \urort{} we have provided
inconclusive findings of the success of such a social navigation technique.
We take our results as indications of the usefulness of activity streams.
Further research is needed to abandon the idea or recommend its usage.
Since we did not find any noticeable negative results towards activity streams
we can recommend implementations or prototypes of this feature in web sites
with similar dynamics as \urort{}.

\section{Future Work}

Our new definition of social navigation in light of the essentialness of peers
were based on cursory observations and more detailed analysis of two social
web sites. We regard our findings of how social navigation is used in social
web sites as early work in this area which needs to be expanded on. More
widespread collection and analysis of social navigation in modern web sites is
needed to see if our observations holds true.

We've described social navigation as a disparate field. We hope our work
some extent can remedy this problem. One venue for further work to make social
navigation a better understood term would be to create a taxonomy of social
navigation types. A set of design patterns for when, where, how, and why these
various types of social navigation should be used could accompanying such a
taxonomy.

\parabreak

In light of the problems we experienced with adopting sufficient number of
experiment participants we see the need for a laboratory study of activity
streams. Had time permitted in our master thesis work its quite plausible that
we had conducted an in-lab experiment where factors as technological ability
of the respondents would not interfere with the generalizability of our study.

There should also be studies conducted of the use of activity streams over a
longer period than 11 days as for our study. Its possible that activity
streams become more useful (or perhaps annoying and intrusive) after prolonged
use. We would also like to see how longer usage periods could have potential
effects on the importance on favorites in the case of the \urort{} web site.
