\chapter{Empirical Study of a Social Navigation Prototype}
\label{chapter:empirical}

\section{Research Problem}

\begin{quote}
  Can social navigation trough activity streams influence
  usage of an established web site?
\end{quote}

This question deals with how a specific social navigation technique, activity
streams, could potentially influence experiment participants' usage of a web
site. We also wanted to question how the experiment participants perceived
activity streams in relation to the quality of their usage experience:

\begin{quote}
  Can social navigation trough activity streams improve users' perceived
  quality of usage of an established web site?
\end{quote}

We also had a more technical research question relating to how one can
conduct experiments on established web sites:

\begin{quote}
  Can prototyping with Greasemonkey be used successfully
  for testing user behavior in an established web site without
  intruding on users not involved in the experiment?
\end{quote}

\section{Research Hypotheses}

%%
%% hypotheses listed here
%%

\section{Methodology}
\label{section:empirical.methodology}

This section outlines the methodology we used for testing our research
hypotheses. We'll dive in to the design of the experiment, how we collected
the data, and how data was analyzed.

\subsection{Experiment design}

\citet[\p{78}]{robson93} describes an experiment as a process where:


\begin{items}
  \item The experimenter assignings subjects to different conditions.
  \item The experimenter manipulates one or more variables.
    These variables are \term{independent variables}.
  \item The experimenter measures the effect of the manipulation of
    the independent variables on other variables. These other
    variables are \term{dependent variables}.
\end{items}

We are conducting a real world experiment, meaning that our subjects
are studied in their natural habitat\dash{}not in a laboratory.
The advantages of a real world experiments are firstly that it's easier to
generalize results to a wider real world population since one does not have
an artificial setting as in laboratories. Secondly real world experiments are
not as prone to gaming by its participants. Lastly it's easier to find willing
subjects in the real world.

Real world experiments have some shortcommings compared to laboratory
experiments. Seemingly most important is the lack of control of various
variables which could interfere with the independent variables.
For more about the merrits and disadvantages of real world experiments
see \citet[\pp{80}{87}]{robson93}.

More specifically we're using a two-group experiment design with a test before
and after the independent variables are manipulated\dash{}a \term{treatment}.
The two groups are:

\begin{items}
  \item A \term{experiment group}. This group are given a treatment by
    manipulating an independent variable.
  \item A \term{control group}. This group are not given a treatment but are
    instead given a \term{placebo} which does not manipulate the independent
    variable.
\end{items}

Using two groups means that we can measure the difference between those that
undervent treatment and those that were given a placebo. This gives us a way
to measure the effect of the treatment.
By using a before and after design we are also able to use pre-post
differences as a basis for measuring the effect of treatment or no
treatment.

\begin{figure}
  \includegraphics{fig_experiment_setup}
  \caption[Experiment Overview]{
    Overview of the various parts of the experiment. The population $n_1$
    are given a pretest. $n_2$ completes the prestest and $t_1$ are given
    an treatment prototype while $p_1$ is given a placebo
    prototype by randomization.
    After one of the two types of prototypes are provided, respondents are
    followed-up to check if they had problems installing the prototype
    software. $n_3$ answers the follow-up questions.
    $t_2$ successfully installed the treatment prototype and $p_2$
    successfully installed the placebo prototype. Both $t_2$ and $p_2$ are
    given a posttest.
    $t_3$ of the treatment population $t_2$ and accordingly $p_3$ of the
    placebo population $p_2$ completes the posttest.
  }
  \label{figure:fig.experiment.setup}
\end{figure}

\subsection{Data collection}

\subsection{Data analysis}

\subsubsection{Outliers}

We went trough the collected data looking for outliers
\dash{}\postquote{rowntree00}{%
  extreme (high or low) values}
We were only concerned with such numerical outliers for ratio values
(age, number of favorites) where it would be meaningful to talk about means.
For ordinal values we were not concerned with such outliers as we operated
with medians for conveying the center of a distribution.
If outliers were found, the value was simply deleted from the sample. We did
this knowing that it can be dangerous to eliminate such values from a sample
\citep[\p{60}]{greene03}.

Another form of outliers can be respondents which respond very
monotonous\dash{}indicating that they just tick off questions without any
thought. In such cases the respondents motivation is to complete the
questionnaire as fast as possible. We did not eliminate such responses from
the sample as it's hard to know the exact motivations of the respondents.
In addition we dont have a particular large set of values for each respondent
to base such analysis on and it would be hard to distinguish such outliers.

\section{Results}

\begin{figure}
  \includegraphics{fig_experiment_mortality}
  \caption[Experiment Mortality]{
    Mortality for the various parts of the experiment.
    The population $n_1$ of 789 \urort{} users were given a pretest and
    a population $n_2$ of 123 completes the pretest.
    By randomization a population $t_1$ of 35 were given a treatment while
    a population $p_1$ of 36 were given a placebo.
    The population $n_2$ were given a follow-up questionnaire for checking
    how the installation of the prototypes went. $n_3$ completed the follow-up
    questionnaire.
    A population $t_2$ of 25 and a population $p_2$ of 20 managed to install
    the treatment and placebo prototype respectively.
    Of those a population $t_3$ of 14 from the treatment population $t_2$ and
    a population $p_3$ of 11 from the placebo population $p_2$ completed
    a posttest.
  }
  \label{figure:fig.experiment.setup}
\end{figure}

\begin{table}[h]
  \begin{tabular}{lrr}

    &
    \multicolumn{1}{c}{Individuals} &
    \multicolumn{1}{c}{\% of respondents} \\

    \cmidrule(lr){2-2}
    \cmidrule(lr){3-3}

    Respondents &
    125 &
    \\

    Completed installation &
    72 &
    57.6 \\

    Used application &
    45 &
    36.0 \\

  \end{tabular}
  \caption[Respondents Falloff]{%
    Respondent falloff during installation and use of our prototype
    application}
  \label{table:respondents.falloff}
\end{table}

\subsection{Pretest}

\subsubsection{Respondent profiles}

\begin{figure}[!ht]
  \begin{tikzpicture}

    % Name      Size   Percent  Value
    % ------------------------------
    % Total      124   100      4
    % Male       114    91.9    3.67741
    % Female     10      8.1    0.32258
    %
    % y1               100      4
    % y2                75      3
    % y3                50      2
    % y4                25      1

    \draw[ycomb,color=bar,line width=1.2cm]
       plot coordinates{(1,3.67741)
                        (3,0.32258)};

    \begin{footnotesize}
      \draw (1, 3.67741) node[anchor=south] {114};
      \draw (3, 0.32258) node[anchor=south] {10};

      \foreach \y / \lines in {4/100,3/75,2/50,1/25}
        \draw (0,\y) node[anchor=mid east] {\lines\%};

      \foreach \x / \library in {1/Male,
                                 3/Female}
        \draw (\x, -0.4) node[anchor=base] {\library};

      \begin{scope}[color=white,line width=1pt]
        \draw (0,1) -- (10,1);
        \draw (0,2) -- (10,2);
        \draw (0,3) -- (10,3);
      \end{scope}
    \end{footnotesize}
  \end{tikzpicture}

  \caption[Gender of Respondents]{%
    The gender of respondents from a total of 124 responses.
  }
  \label{figure:chart.respondents.gender}
\end{figure}

\subsection{Post-installation test}

\subsection{Posttest}

\section{Discussion}

\subsection{Activity streams as a social navigation technique}

\subsection{Transparent prototyping with Greasemonkey}

During our development of a prototype application with Greasemonkey for
enhancing an established web page we got a feel for its pros and cons from a
development perspective.

\subsubsection{Requires no access to the established implementation}

\subsubsection{Requires little knowledge of the established implementation}

\subsubsection{Requires more work than altering the established
  implementation}

\subsubsection{Fragile when the established implementation is changed}
% Only happened once during a two month span on \urort{}. What was scary
% thought was that the changes made the user script on the client side
% obsolete. If this had happened under production usage when the user scripts
% was pushed to the clients we would be in a world of trouble. Changes on the
% server side platform can be handled more transparently.

\subsubsection{Less performant than the established implementation}

\parabreak

When we conducted a study of our prototype with real world users
we got valuable feedback on how well such a system works for the average user.

\subsubsection{Limited in browser selection}

\subsubsection{Difficulties with installing Greasemonkey}

\subsubsection{Difficulties with installing user-scripts}

\section{Generalizability and Validity}
