\chapter{Empirical Study of a Social Navigation Prototype}
\label{chapter:empirical}

\section{Research Problem}

\begin{quote}
  Can social navigation trough activity streams influence
  usage of an established web site?
\end{quote}

This question deals with how a specific social navigation technique, activity
streams, could potentially influence experiment participants' usage of a web
site. We also wanted to question how the experiment participants perceived
activity streams in relation to the quality of their usage experience:

\begin{quote}
  Can social navigation trough activity streams improve users' perceived
  quality of usage of an established web site?
\end{quote}

We also had a more technical research question relating to how one can
conduct experiments on established web sites, hoping to answer the previous
research questions:

\begin{quote}
  Can prototyping with Greasemonkey be used successfully
  for testing user behavior in an established web site without
  intruding on users not involved in the experiment?
\end{quote}

\section{Research Hypotheses}

\section{Methodology}
\label{section:empirical.methodology}

\begin{figure}
  \includegraphics{fig_experiment_setup}
  \caption[Experiment Overview]{
    Overview of the various parts of the experiment. The population $n_1$
    are given a pretest. $n_2$ completes the prestest and half are given
    an treatment prototype while the other half is given a placebo
    prototype. After one of the two types of prototypes
    are provided, respondents are followed-up to check
    if they had problems installing the prototype software. $n_3$ answers
    the follow-up questions.
    $n_4$ successfully installed the prototype and are given a posttest.
    $n_5$ completes the posttest.
  }
  \label{figure:fig.experiment.setup}
\end{figure}

\section{Results}

\begin{table}[h]
  \begin{tabular}{lrr}

    &
    \multicolumn{1}{c}{Individuals} &
    \multicolumn{1}{c}{\% of respondents} \\

    \cmidrule(lr){2-2}
    \cmidrule(lr){3-3}

    Respondents &
    125 &
    \\

    Completed installation &
    72 &
    57.6 \\

    Used application &
    45 &
    36.0 \\

  \end{tabular}
  \caption[Respondents Falloff]{%
    Respondent falloff during installation and use of our prototype
    application}
  \label{table:respondents.falloff}
\end{table}

\subsection{Pretest}

\subsubsection{Respondent profiles}

\begin{figure}[!ht]
  \begin{tikzpicture}

    % Name      Size   Percent  Value
    % ------------------------------
    % Total      124   100      4
    % Male       114    91.9    3.67741
    % Female     10      8.1    0.32258
    %
    % y1               100      4
    % y2                75      3
    % y3                50      2
    % y4                25      1

    \draw[ycomb,color=bar,line width=1.2cm]
       plot coordinates{(1,3.67741)
                        (3,0.32258)};

    \begin{footnotesize}
      \draw (1, 3.67741) node[anchor=south] {114};
      \draw (3, 0.32258) node[anchor=south] {10};

      \foreach \y / \lines in {4/100,3/75,2/50,1/25}
        \draw (0,\y) node[anchor=mid east] {\lines\%};

      \foreach \x / \library in {1/Male,
                                 3/Female}
        \draw (\x, -0.4) node[anchor=base] {\library};

      \begin{scope}[color=white,line width=1pt]
        \draw (0,1) -- (10,1);
        \draw (0,2) -- (10,2);
        \draw (0,3) -- (10,3);
      \end{scope}
    \end{footnotesize}
  \end{tikzpicture}

  \caption[Gender of Respondents]{%
    The gender of respondents from a total of 124 responses.
  }
  \label{figure:chart.respondents.gender}
\end{figure}

\subsection{Post-installation test}

\subsection{Posttest}

\section{Discussion}

\subsection{Activity streams as a social navigation technique}

\subsection{Transparent prototyping with Greasemonkey}

During our development of a prototype application with Greasemonkey for
enhancing an established web page we got a feel for its pros and cons from a
development perspective.

\subsubsection{Requires no access to the established implementation}

\subsubsection{Requires little knowledge of the established implementation}

\subsubsection{Requires more work than altering the established
  implementation}

\subsubsection{Fragile when the established implementation is changed}
% Only happened once during a two month span on \urort{}. What was scary
% thought was that the changes made the user script on the client side
% obsolete. If this had happened under production usage when the user scripts
% was pushed to the clients we would be in a world of trouble. Changes on the
% server side platform can be handled more transparently.

\subsubsection{Less performant than the established implementation}

\parabreak

When we conducted a study of our prototype with real world users
we got valuable feedback on how well such a system works for the average user.

\subsubsection{Limited in browser selection}

\subsubsection{Difficulties with installing Greasemonkey}

\subsubsection{Difficulties with installing user-scripts}
