\chapter{Methodology}
\label{chapter:methodology}

This chaper will include information in how data was collected. So far this
includes content inventory/analysis.

\section{Content Analysis}

The term \emph{content analysis} is most often used to signify a research
technique used in the social sciences.
\citet[p.~18]{krippendorff03} defines it as:
``a research technique for making replicable and valid
inferences from texts (or other meaningful matter) to the contexts of their
use''.

This is however not the use of the term we're concerned with here. We consult
content analysis as the more pragramtic practice conducted within the field of
\emph{information architecture}%
\sidenote{
  \citet[p.~4]{morville06} defines information architecture as:
  \begin{inparaenum}[(a)]
    \item the structural design of shared information environments,
    \item the combination of organization, labeling, search, and navigation
      systems within web sites and intranets,
    \item the art and science of shaping information products and experiences
      to support usability and findability, and
    \item an emerging discipline and community of practice focused on bringing
      principles of design and architecture to the digital landscape.
  \end{inparaenum}
}.
Content analysis is deployed as a technique by information architects for
helping them generate a sound and well structured web site architecture.
In it's essence a content analysis should identify the various
relationships (or lack of correlation) between a web site's content items.
It consists of two phases:
\begin{inparaenum}[(i)]
  \item a collection of a representative sample of data and
  \item an analyses of this collected data
\end{inparaenum}
\citep[pp.~241--243]{morville06}.

\subsection{Inventory}

A \emph{content inventory} is a technique for collecting data from web sites
in a structured manner. It's strength as a technique lies in it's ability of
truly informing it's receipents about a web site's content. Content
inventories are often tedious and time consuming. \citet[p.~267]{wodtke02}
argues that every single bit of content needs to be determined while
\citet{morville06} believes a representative sample could be sufficient.

The web sites that are interesting to look at in our research are vast and
loaded with enourmous ammounts of user generated content. An all-inclusive
approach to content gethering would simply be impossible in such situations.
As a remedy to this we've decided to ignore certain parts of web sites in our
content inventories since the scope of our research is limited to navigational
constructs and only those wich have a social nature.

Our experience is that social navigation and more static navigation are
intermixed all over web sites. Often one have to use non-social forms of
navigation before social navigational options appear. Thus we could not simply
ignore navigational aims wich were non-social in our content inventory phase.
We did however eliminate the following parts of web sites:

\begin{description}
  \item[something] description of it
\end{description}

In addition to eliminating certain form of web pages we synthesised abstract
page representations by introducing variables. %finish


\subsection{Analysis}

\subsection{Samping}


Instead of using content analysis as a means for improving on an existing
site's content architecture we'll be tailoring this technique to best help us
discover and understand social navigation patterns in infamous web sites which
are known to make good use of such navigational designs.

