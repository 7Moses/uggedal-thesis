\chapter{Background}
\label{chapter:background}

% 1. how the literature was collected (describe it pragmatically)
%
% 2. literature review (summary, analysis, and comparisons)
%
%   A literature review should answer:
%
%     * What do we already know about the topic?
%     * What do you have to say critically about what is already known?
%     * Has anyone else done anything exactly the same?
%     * Has anyone else done anything that is related?
%     * Where does your work fit in with what is done before?
%     * Why is your research worth doing in the light of what has
%       already been done?
%
%   A literature review should be a dialodic rather than a mere
%   replication of other peoples writing. Should not be a laundry
%   list of previous studies.
%
%   Be focused and critical. Include an incisive critique that will help your
%   peers see the world differently.
%
% 3. introduction to terms as folksonomy, tagging, geotagging, etc
%
% 4. paragraph or two about my subject related to popular literature
%    (search Amazon or Library of Congress and say something like: there
%    were X books about this subject, the first was published in 2001
%    but the majority of books were published the last two years, and
%    maybe show a graph)

\section{Literature Search}

Before a literature search was conducted we did some preliminary thinking
about
\begin{inparaenum}[(a)]
  \item the focus of our topic to get more precise results, and
  \item what literature databases would yield sufficient and accurate
    findings.
\end{inparaenum}
Based on these concerns we settled on the literature indexes layed out in
Table~\ref{table:literature.databases}
(p.~\pageref{table:literature.databases}) and used the following keywords%
\sidenote{
  With warying use of modifiers (i.e. AND) or quotations to find exact phrases
}
for seach:

\begin{description}
  \item[social navigation] is the concept of our main topic.
  \item[collaborative filtering] is often used to realize our main topic.
  \item[recommender system] can be an application of our main topic.
  \item[tagging] can be related to our topic depending on use.
\end{description}

\sidetable{Literature Databases}{%
  \label{table:literature.databases}
  \begin{tabular}{p{0.6\marginparwidth}l}

    \toprule
    Name & Type \\
    \midrule

    ACM Digital Library &
    Fulltext \\

    The Collection of Computer Science Bibliographies &
    Bibliography \\

    Inspec Online &
    Index \\

    HCI Bibliography &
    Bibliography \\

  \end{tabular}
}

In addition to keyword based search we also conducted citation searches on the
articles that in our oppinion seemed to be the most important in the field.
The articles that we found relevant during our literature search phase was
collected and studied. During this process we eliminated articles by the same
authors where similar topics and implementations were discussed and focused on
either the most recent or the most representative article.

\section{Litterature Review}

First we'll concentrate on the research where social navigation is used
conciously as a concept. By this we mean the research where either social
navigation is defined, redefined or problems relating to the concept is
discussed with a basis in such definitions. Afterwords we'll look at topics
which we believe can be included in the discussion of social navigation or are
closely related. Some of the research that has been conducted in the space of
social navigation and related areas does not share our focus on the Web.
We still found much such research interesting in spite of their attention to
generalized problems or specific problems in other fields than the Web.

\subsection{Social Navigation}
\label{section:background.social.navigation}
% Could we say that the ideas of social navigation was put forth by Bush?

The term social navigation was introduced by \citet{dourish94} in a short
article where they discussed three types of navigational mechanisms, spatial,
semantic, and social, wich they argue can be seperated even though there is
evidence of situations where the different mechanisms are combined.
In their description of the social type they coined the term
\emph{social navigation}:

\begin{quote}
  When navigable information systems are extended to support collaborative
  activity, a third model of navigation arises. This is \emph{social}
  navigation. In social navigation, movement from one item to another is
  provoked as an artifact of the activity of another or a group of others.
  \citep[p.~1]{dourish94}
\end{quote}

\citeauthor{dourish94} examplifies two cases where neither location
(spatial) nor content (semantic) is used for exploration--the social model
is used on it's own. The first example is of home pages where the creators
have authored a list of web pages they find interesting and thereby creating
oppertunity for navigation based on social factors. Their second example
is of the Tapestry \citep{goldberg92} system where electronic messages can be
voted upon and users can use such metrics for navigating items that seem to be
of interest. Such a technique is called \emph{collaborative filtering},
a term wich will be discussed later as a concept related to social navigation.
Based on these two experiences \citeauthor{dourish94} argues that we possibly
need to move away from spatial models of navigation and rather focus on
designing explicitly with semantic and social navigational techniques.
