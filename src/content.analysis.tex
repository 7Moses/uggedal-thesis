\chapter{Content Analysis}

Content analysis is a technique deployed by information architects for helping
them generate a sound and wellstructured website architecture. It consists of
two phases: collection of a representative sample of data and an analysis of
this content. In it's essence a content analysis should indentify the various
relationships (or lack of correlation) between a website's content items.

Instead of using content analysis as a means for improving on an existing
site's content architecture I'll be tailoring this technique to best help me
discover and understand social navigation patterns in infamous websites which
are known to make good use of such navigational designs. This means that I'll
concentrate on only core content objects and relationships amongs them
which are organically generated -- relationships which are made as part of
past users behaviour that can be leveraged by other users as a social form
of navigation.

A high-level mapping of the content and it's various relationships
will be carried out after having performed the low-level content inventory
and the subsequent analysis of these findings.

\section{Flickr}

Flickr is a photo sharing site which are known to be on the cutting edge when
it comes to enabling new and innovating navigational features. This subsequent
analysis of Flickr will be carried out as a registered user. One has to be
registered for interacting with the site in such a way that one leaves
persistent traces even though all content is available and open for all users.

\section{Amazon}

\section{Facebook}

\section{Del.icio.us}
