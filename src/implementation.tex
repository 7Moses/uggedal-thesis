\chapter{Implementation}
\label{chapter:implementation}

% This chapter should include design choices for my implementation. For
% example choices taken for generating relevant data for test users
% and a computational sound method for doing so. As JavaScript in browsers are
% quite inefficient it will probably be necessary to persist data at a server
% in some sort of cache. The clients could then get this data by invoking a
% single request. The result could for instance be JSON serialized. Scraping
% of such data is probably done more efficient and safer at the server side
% since multiple XMLHttpRequests in the clients for scraping and parsing could
% prove to be quite computational expensive.

As we've seen in 
\sectionref{building.on.top.of.the.web}
it's possible to build applications on top of existing web sites by creating
transparent prototype implementations. This chapter starts with an account of
what kind of navigation system we wanted to build, goes on to describe why we
decided on such navigational designs, and concludes with an explanation of how
the implementation was built\dash{}the ingredients of our implementation.

\section{Design}

% no new data. making existing data more readily available.

\section{Process}
% Prototype, exploratory
% TDD, BDD

\section{Architecture}

Our implementation basically needs to do two things:

\begin{enum}
  \item Collect existing data from various places on the \urort web site.
  \item Display this data in existing web pages on the \urort web site in
    a way that we hope will enhance navigation.
\end{enum}

We decided to use a server--client architecture so that we could offload some
of the more computationally expensive operations off the client and onto a
dedicated server. Another benefit of such an architecture is that it allows us
to cache data globally\dash{}shared by all clients. Therefore data collection
is handled on the server side, while data display obviously is handled on
the client side.

Based on the design decisions we've taken we needed to figure out what parts
our software architecture should be composed of. 
Firstly one important aspect with the software stack used in this
implementation\dash{}from operating system to third party libraries\dash{}is
that it should only consisting of open source software. In our experience it's
invaluable to have sources available for all involved software. If one
encounters abnormal behavior or bugs it's much easier to locate them when one
have sources available and one can trivially (depending on the complexity of
the problem) create a patch that sorts them out.

\subsection{Client Side}

\subsubsection{Platform}
\label{section:implementation.architecture.client.side.platform}

The platform for the clients is in essence a web browser. We are making
changes to a web page after all. The web browser have to be explicitly chosen
to be one that readily supports scripting existing
web pages\dash{}a term often called \term{user scripting}.
The \project{Firefox}%
\sidenote{
  Available at \url{http://www.mozilla.com/en-US/firefox}.
}
web browser was the first browser providing a
plug-in for handling such scripting of web pages and seems to have the most
mature implementation in our view. Since Firefox also is the
most adopted%
\sidenote{
  According to \citet{onestat08} Firefox was the second most used web browser
  only surpassed by \project{Internet Explorer}.
}
cross-platform open source web browser the platform choice was quite easy.

Firefox provide user scripting trough the means of the
\project{Greasemonkey}%
\sidenote{
  Available at \url{http://greasespot.net}.
}
browser extension. Essentially all it provides is the ability for a user to
install a script which can manipulate the contents of a web page
in various ways using the \term{\abbr{DOM}}%
\sidenote{
  The \abbr{DOM} is a three of objects representing the hierarchical
  structure of nested tags (with text and attributes) in \abbr{HTML}
  documents \citep[pp.~307--310]{flanagan06}.
}.
When a user have such a script for a specific web page installed it's
instructions will be executed on the next visit to the given site, enabling
all kind of modifications possible in the \abbr{DOM}. This is the ingredient
that enables us to display new information trough navigational designs on the
\urort web site.

Although we've settled on the Firefox and Greasemonkey platform there is a
certain possibility that our implementation could work in other browsers
providing user scripting. The \project{Opera} browser provides user scripting
without any plugins%
\sidenote{
  For more information see
  \url{http://www.opera.com/support/tutorials/userjs/examples}.
},
the \project{Safari} browser can handle user script with the
\project{GreaseKit}%
\sidenote{
  Available at \url{http://8-p.info/greasekit}.
}
plug-in. We have not tested such support for ourselves since we decided to
focus all our efforts on one platform.

\subsubsection{Programming Language}

The ability to programmatically alter behavior inside web browsers was first
introduced by \project{Netscape} in their \val{2.0} version of the web browser
with the same name. \project{JavaScript} was first intended to be a
lightweight scripting language for gluing together \abbr{HTML} and applets
written in the \project{Java} programming language \citep{netscape95}. Java
applets never took of and JavaScript soon became the \latin{de facto} standard
for enabling behavior on the Web and was standardized as
\project{ECMAScript} in 1997 \citep{ecma99}.

Because of this we had no say in what programming language to use on the
client side. That is not to say that JavaScript is a poor programming
language. Contradictory to it's name, JavaScript bears few similarities to the
Java language.%
\sidenote{
  The name was more of a marketing decision when Netscape teamed up
  with Sun (the maker of Java) before it's initial release
  \citep[p.~2]{flanagan06}.
}
Despite it's origins as a scripting language JavaScript is now considered
a full-featured modern programming language
(\citealp[p.~2]{flanagan06};
\citealp[p.~3]{resig06}).

%%% write more about JavaScript features as a modern language from pro
%%% JavaScript tech.
% list it's good features, and that recently the style of JavaScript has
% changed.

\subsubsection{Convenience Library}

We decided to use a JavaScript library to make interactions with the
\abbr{DOM} simpler.
In addition there recent JavaScript convenience libraries provide a
unified interface to the browser\dash{}abstracting away inconsistencies
between browser vendors. Lately a myriad of such frameworks have appeared,
but the most interesting ones seems to be
\project{Prototype},
\project{Yahoo! UI Library} (\abbr{YUI} for short),
\project{MooTools},
\project{MochiKit}, and
\project{jQuery}.%
\sidenote{
  Available, in respective order, at
  \url{http://www.prototypejs.org},
  \url{http://developer.yahoo.com/yui},
  \url{http://mootools.net},
  \url{http://mochikit.com}, and
  \url{http://jquery.com}.
}
There are other frameworks available that provide everything but the kitchen
sink but we needed a lightweight or modular solution.

As can be seen in the following table we summarized the size of the most
current version for each library of this writing. These are not exact
metrics\dash{}we selected not to include certain widgets and logging
facilities for the modularized libraries\dash{}but should provide clear
guidance. To keep a level playing feel in this comparison we did not use
minified (removal of comments and unnecessary spaces) or packaged (compressed)
versions of the libraries. All comments and documentation was stripped with a
small script presented in \sourcecodepageref{javascript.comment.sripping}
since the in-line documentation and commenting varied amongst the libraries.

\sidetable{JavaScript Library Comparison
           \label{table:javascript.library.comparision}}{%
  
  \begin{tabular}{lr}

    Library & Size \\
            & (kB) \\
    \midrule

    MooTools (1.11)     & 67 \\
    jQuery (1.2.3)      & 91 \\
    Prototype (1.6.0.2) & 122 \\
    MochiKit (1.3.1)    & 181 \\
    \abbr{YUI} (2.5.0)  & 286 \\

  \end{tabular}
}

We played arround a bit with the different libraries to get a feel for how
they worked. What follows is a comparison of simple \abbr{DOM} manipulation
for the different libraries. We followed the official documentation for the
various libraries and tried to solve or problem as succinct and clearly as
possible. We tried to add a \code{class} attribute of \val{highlight} to
all \code{em} elements with an descendant \code{p} element: 

\begin{scode}{Java}{dom.manipulation.prototype}{%
  \abbr{DOM} Manipulation with Prototype}{%
  DOM manipulation in JavaScript with the Prototype library}
\begin{lstlisting}
getElementsBySelector("p em").each(function(em) {
  em.addClassName("highlight");
});
\end{lstlisting}
\end{scode}

\begin{scode}{Java}{dom.manipulation.yui}{%
  \abbr{DOM} Manipulation with Yahoo! UI Library}{%
  DOM manipulation in JavaScript with the Yahoo! UI library}
\begin{lstlisting}
var em = YAHOO.util.Selector.query("p em"); 
YAHOO.util.Dom.setClass(em, "highlight);
\end{lstlisting}
\end{scode}

\begin{scode}{Java}{dom.manipulation.mootools}{%
  \abbr{DOM} Manipulation with MooTools}{%
  DOM manipulation in JavaScript with the MooTools library}
\begin{lstlisting}
$$("p em").each(function(em){
  em.addClass("highlight");
});
\end{lstlisting}
\end{scode}

\begin{scode}{Java}{dom.manipulation.mochikit}{%
  \abbr{DOM} Manipulation with MochiKit}{%
  DOM manipulation in JavaScript with the MochiKit library}
\begin{lstlisting}
var p = getElementsByTagAndClassName("p");
for (i = 0; i < p.length; i++) {
  em = getElementsByTagAndClassName("em","*", p[i]);
  for (j = 0; j < em.length; j++) {
    addElementClass(em, "highlight");
  }
}
\end{lstlisting}
\end{scode}

\begin{scode}{Java}{dom.manipulation.jquery}{%
  \abbr{DOM} Manipulation with jQuery}{%
  DOM manipulation in JavaScript with the jQuery library}
\begin{lstlisting}
$("p em").addClass("highlight");
\end{lstlisting}
\end{scode}

When we compare these rather trivial problem solutions it becomes apparent
that choosing a JavaScript library can have major impact on how easily
implemented and understood your code will be. Four of the five libraries
have support for selector syntax based on
that found in\abbr{CSS}%
\sidenote{
  Cascading Style Sheets. A stylesheet language for describing the
  presentation of for instance \abbr{HTML} documents.
}.
This is what makes the MochiKit example the most complex one, requiring the
developer to do two queries into the \abbr{DOM} and construct two
loop structures for iterating over the results.
Prototype and MooTools also requires the developer to loop over a single
result set, but the iteration is abstracted into an \code{each} function
making the logic a bit more clearer. Yahoo! UI Library's \abbr{DOM} functions
works on both single elements and collections of elements\dash{}eliminating
the need for an explicit loop structure. Notice though that the library from
Yahoo! relies heavily on namespacing\dash{}which is a good thing for
interoperability with other libraries\dash{}but can be a bit verbose at times.

Going even a bit further in clearity is the solution written with jQuery.
Every query into the \abbr{DOM} returns a special jQuery object which means
that one can call methods like \code{addClass} directly on this object
regardless if the jQuery object holds a single or multiple elements.
Also unique to jQuery is the fact that every method call returns a new jQuery
object. This means that one can \term{chain} methods together, expressing
succinctly and clearly what you intend to accomplish with your code. We can
extend our initial problem and add some punctation inside our \code{em}
element:

\begin{scode}{Java}{jquery.method.chaining}{%
  jQuery Method Chaining}{%
  Chaining multiple methods together in jQuery}
\begin{lstlisting}
$("p em").addClass("highlight").append("!");
\end{lstlisting}
\end{scode}

Based on the minimal file size jQuery showed (only outbested by MooTools) and
it's clear and unique syntax we selected it as the JavaScript library for our
implementation. It seems others have take jQuery and it's virtues to hart as
many large corporations like Google, Intel, Dell, and BBC have used it in
their public facing offerings.%
\sidenote{
  For a complete list see \url{http://docs.jquery.com/Sites_Using_jQuery}.
}

\subsection{Server Side}

\subsubsection{Platform}

\subsubsection{Programming Language}

When doing prototype work it's important that the programming language one
uses is efficient to work with. This means that programmer efficiency is more
important than computational efficiency (a language's native performance).
Since we didn't have time to invest in learning a new language we had to do
with those we knew from before. Of those \project{Ruby}%
\sidenote{
  Ruby recedes at \url{http://ruby-lang.org}.
},
\project{Python}%
\sidenote{
  The home of Python is \url{http://python.org}.
}, and
\project{Common Lisp}%
\sidenote{
  Common Lisp, the prevalent Lisp dialect today, is a standard \citep{ansi96}
  and has many implementations.
  A gateway to this language and it's many implementations can be found
  at \url{http://common-lisp.net}.
}
were the ones with language features that fitted our development process.

They are all \term{latent typed}%
\sidenote{
  Latent typing ``is a style of typing that does not require (or perhaps even
  offer) explicit type declarations''\citep{wikipedia08}.
}
and have quite expressive syntax. This makes for concise source code.
\citet{yegge07} argues that the worst thing that can happens to a code base is
size which often is the result of code bloat. In addition, both Ruby, Python,
and Common Lisp are \term{interpreted} languages. This means that the
programmer don't have to go trough a compilation process before he can see the
results of his labor. When prototyping rapidly it's quite convenient to make
small changes and see the results instantanously.

% discuss them against eachother here



As it turns out, the most important criteria for choosing the implementation
language was it's library support. In the next section we discuss our options
of such libraries or frameworks. Based on our findings there and the
previous server side language discussion we landed on Ruby.

\subsubsection{Data Extraction Library}

The core library we need is one that handles data extraction from existing
web pages, so called \abbr{HTML} \term{scraping}. While it's possible to
handle such problems with regular expressions, this becomes tedious after a
while. We therefore prefer a special purpose library.

The major deciding factor when we selected the implementation language was
the availability of such a library and it's usefulness. Since we've already
revealed Ruby as our implementation language we're killing the suspense.
Our data extraction library is called \project{Hpricot}%
\sidenote{
  Hpricot can be obtained from \url{http://code.whytheluckystiff.net/hpricot}.
  A curious note: Hpricot is written by the same person who created
  Hoodwink.d\dash{}our inspiration for a transparent prototype implementation.
}
and makes \abbr{HTML} parsing almost a fun endeavor in our opinion.

The Python alternative for web page scraping is \project{Beautiful Soup}.
We were not able to find any libraries specially made for \abbr{HTML} scraping
implemented in Common Lisp. There exists several \term{\abbr{XML}}%
\sidenote{
  Extensible Markup Language. General purpose markup language specification
  that enables implementors to create custom markup languages.
  \abbr{HTML} is not a subset (specified in) \abbr{XML}.
  \term{\abbr{XHTML}} on the other hand, a reformulated version of
  \abbr{HTML}, is a subset of \abbr{XML}.
%%% cite!
}
libraries that could handle our tasks, but none as well integrated as
the Ruby and Python options.

To get a feel for the difference between Hpricot and Beautiful Soup we tried
them out on some trivial examples. Under you'll see the listings for one
of these examples. We are trying to find an \code{em} element with a class
of \code{citation}, which have a \code{p} element as it's parent,
in a \abbr{HTML} document contained in the \code{html} object:

\begin{scode}{Python}{parsing.beautiful.soup}{%
  Parsing with Beautiful Soup}{%
  HTML parsing in Python with Beautiful Soup}
\begin{lstlisting}
html('p').content.findNextSiblings('em', 'citation')
\end{lstlisting}
\end{scode}

\begin{scode}{Ruby}{parsing.hpricot}{%
  Parsing with Hpricot}{%
  HTML parsing in Ruby with Hpricot}
\begin{lstlisting}
  html/'p > em.someclass'
\end{lstlisting}
\end{scode}

We feel that Hpricot's syntax is much clearer than that of Beautiful Soup.
This could be a personal preference since we've used \abbr{CSS} for a long
time and Hpricot's selector syntax is based on \abbr{CSS} and Xpath, just as
jQuery. Hpricot was in fact initially based on jQuery's selector syntax
\citep{why06}. This means that we can use the same syntax for selectors on the
server and client side\emph{}a cognitive advantage.

\subsubsection{Data Fetching Library}

Since we've selected Ruby as our development language of choice
\executable{open-uri}, part of the standard Ruby library, is the way to fetch
documents over \abbr{HTTP}%
\sidenote{
  Hyper Text Transfer Protocol. The data transfer protocol for the Web.
}.
\executable{open-uri} is trvial to use and integrates nicely with Hpricot:

\begin{scode}{Ruby}{fetching.openuri.parsing.hpricot}{%
  Fetching and Parsing with Hpricot and open-uri}{%
  Fetching a HTML document with \executable{open-uri}
  and parsing it with Hpricot to find the first and
  last name of a hCard Microformat}
\begin{lstlisting}
require 'hpricot'
require 'open-uri'

html = Hpricot(open('http://redflavor.com'))
(html/'address.vcard > .fn').inner_html
# => "Eivind Uggedal"
\end{lstlisting}
\end{scode}


\subsubsection{Web Framework}

A web framework or rather \abbr{HTTP}
framework is needed to make the
collected data available for our client.

% modest needs. serve JSON. maybe include sub sub section before this one
% with selection of a JSON serializer/library.


\section{Development Tools}

As with the implementation platforms, languages, and third party libraries
our first criterion for selecting development tools is freedom.

\subsection{Version Control}

We've found it indispensable to use \term{version control} when writing code
and even used it when authoring this thesis. We'll not spend time to discuss
the merits of version control since we feel it's benefits are major and
using one induces almost zero overhead in your working process. Sometimes we
feel that the use of version control can guide you when conducting complex
tasks.

There are however several different forms of version control system one can
use. One of the most used version control implementations the last years
in open source circles was
\project{Subversion}%
\sidenote{
  Available at \url{http://subversion.tigris.org}.
}\dash{}a \term{centralized version control system} meaning that one central
server holds the version controlled code repository and it's history.%
\sidenote{
  Developers on the client side have working copies and need to contact the
  centralized server to get a hold of historical data and create new history.
}
Recently \term{decentralized version control systems} have become more popular
amongst developers. A decentralized model means that every developer can have
their own repository consisting of all history.%
\sidenote{
  You can for instance be without internet connectivity and still commit
  changes, revert to previous versions, and handle all other tasks your
  version control system supports.
}
Code is then shared either in a push or pull fashion between such individual
repositories. This enables a much better model for collaboration.
We favor this last model of version control and so have projects
like \project{Linux}, \project{X}, \project{Mozilla},
and \project{OpenSolaris}.%
\sidenote{
  \citeauthor{torvalds07}, author of the Linux kernel,
  have described Subversion and centralized version control
  as fundamentally flawed since it's supposed to be a
  \q{\project{CVS} done right}. Since he feels CVS is flawed Subversion
  is therefore inherently flawed \citeyearpar{torvalds07}.
}

Based on criteria of performance and current adoption there are in our view
only two interesting decentralized version control systems:
\project{Git}%
\sidenote{
  Available at \url{http://git.or.cz}.
}
and \project{Mercurial}%
\sidenote{
  Available at \url{http://www.selenic.com/mercurial}.
}. Both are unique in that they don't track meta-data, they just track
content and meta-data are thereby inferred from the content.
At a very high level view Mercurial have a better user interface and Git
supports some advanced features the former don't have. We opted to used
Mercurial for this development project since we've substantial experience in
using it and did not need any of Git's advanced features.

\subsection{Editor}

A developer's main tool for authoring software is his editor. Sometimes the
language of implementation warrants a specialized editor with aids for
handling cumbersome tasks specific to that language. Such an editor is often
called an \abbr{IDE}%
\sidenote{
  A good example of an \abbr{IDE} (integrated development environment
  for short) is \project{Eclipse} (available at \url{http://eclipse.org}).
  It was first used for Java development but since extended with
  plugins for handling other programming languages and families.
}
and are used most often for languages like Java and C\#.
\citet{murphy06} found that developers mostly use an \abbr{IDE} for navigating
large collections of source code, refactoring code, debugging code, and
interacting with revision control systems in addition to normal editor usage.
Development environments found in Lisp%
\sidenote{
  \prequote[p.~69]{sandewall78}{%
    describes the nature and benefits of the Lisp environment as}{%
      The `residential' design of programing systems, whereby all facilities
      for the user are integrated into one system with which the user
      communicates during the entire interactive session, offers great
      possibilities for user convenience}
}
and Smalltalk%
\sidenote{
  Similar to Lisp's programming environment
  \postquote[p.viii]{goldberg83}{%
    Smalltalk is designed so that every component in the system that is
    accessible to the user can be presented in a meaningful way for
    observation and manipulation}

}
are surpassing \abbr{IDE} types in integration and interactiveness even though
they preceded them.

The programming languages we previously settled on, JavaScript and Ruby,
are very expressive and dynamic in their nature in addition to being
interpreted instead of compiled. Our experience is that \abbr{IDE} usage for
such languages stands more in the way than aid you as a programmer during
your problem solving process.
\citet{bray07} conducted a rather unscientific survey of 1000 Ruby
programmers. Despite of the surveys shortcomings it showed that
the majority of Ruby programmers used non-\abbr{IDE} editors for their
development.

The interactive experience provided by Lisp and Smalltalk implementations are
sadly missing%
\sidenote{
  Ruby has an interactive interpreter similar to those found in Lisp and
  Smalltalk environments called \executable{irb}. It's not integrated into an
  overall programming environment and therefore is mostly used for testing out
  small ideas.
}
from JavaScript and Ruby implementations. This means that we're left with
finding a good editor which enables us to focus on writing code as efficiently
and safely as possible. Editor selection is highly a matter of preference and
finding one that matches your work process. Powerful editors have a
reputation of being quite hard to learn. But if you get over the steep
learning curve the benefits the editor gives you are worth it.
\citeauthor{orenstein08} have experienced how much effort programmers can
invest in something seemingly trivial as an editor:

\begin{citequote}{orenstein08}
  If the thought of switching editors doesn't fill you with quite a bit of
  dread, what you're using now is almost certainly under powered, and you
  definitely haven't customized it enough.
\end{citequote}


