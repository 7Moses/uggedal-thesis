\section{Method}
\label{section:empirical.methodology}

This section outlines the methodology we used for testing our research
hypotheses. We'll dive in to the design of the experiment, how we collected
the data, and how data was analyzed.

\subsection{Experiment design}
\label{section:empirical.methodology.experiment.design}

\citet[\p{78}]{robson93} describes an experiment as a process where:


\begin{items}
  \item The experimenter assigns subjects to different conditions.
  \item The experimenter manipulates one or more variables.
    These variables are \term{independent variables}.
  \item The experimenter measures the effect of the manipulation of
    the independent variables on other variables. These other
    variables are \term{dependent variables}.
\end{items}

We are conducting a real world experiment, meaning that our subjects
are studied in their natural habitat\dash{}not in a laboratory.
The advantages of a real world experiments are firstly that it's easier to
generalize results to a wider real world population since one does not have
an artificial setting as in laboratories. Secondly real world experiments are
not as prone to gaming by its participants. Lastly it's easier to find willing
subjects in the real world.

Real world experiments have some shortcomings compared to laboratory
experiments. Seemingly most important is the lack of control of various
variables which could interfere with the independent variables.
For more about the merits and disadvantages of real world experiments
see \citet[\pp{80}{87}]{robson93}.

More specifically we're using a two-group experiment design with a test before
and after the independent variables are manipulated\dash{}a \term{treatment}.
The two groups are:

\begin{items}
  \item A \term{experiment group}: $E$. This group are given a treatment by
    manipulating an independent variable.
  \item A \term{control group}: $C$. This group are not given a treatment but
    are instead given a \term{placebo} which does not manipulate the
    independent variable.
\end{items}

Using two groups means that we can measure the difference between those that
underwent treatment and those that were given a placebo. This gives us a way
to measure any difference induced by the actual treatment since we can detect
possible \term{placebo effects} or \term{observer effects}.
We will then be able to see if respondents are answering positively simply
because they are given something new or they know that they are observed.

By using a before and after design we are also able to use pre-post
differences as a basis for measuring the effect of treatment or no
treatment.



In our case the treatment is analogous with the prototype implementation with
an activity stream for \urort{} as described in
\sectionref{implementation.design.activity.stream}.
The placebo on the other hand is the 
prototype implementation with a favorite list as seen in
\sectionref{implementation.design.favorite.list}.
The favorite list is an integrated part of the treatment implementation
meaning that the only difference between the two prototype versions are
the activity stream\dash{}the independent variable we as experimenters are
manipulating (by introducing the feature).

Bearing in mind the details of our experiment design,
\figureref{fig.experiment.setup} will give an overview of how the experiment
process will be carried out. In addition to the pretest and posttest we
conducted a follow-up survey of our respondents to gauge whether they managed
to install our prototype implementation.

\begin{figure}
  \includegraphics{fig_experiment_setup}
  \caption[Experiment Overview]{
    Overview of the various parts of the experiment. The sample $N$
    are given a pretest. $n_1$ completes the pretest and $E_1$ are given
    an treatment prototype while $C_1$ is given a placebo
    prototype by randomization.
    After one of the two types of prototypes are provided, respondents are
    followed-up to check if they had problems installing the prototype
    software. $n_2$ answers the follow-up questions.
    $E_2$ successfully installed the treatment prototype and $C_2$
    successfully installed the placebo prototype. Both $E_2$ and $C_2$ are
    given a posttest.
    $E_3$ of the treatment sample $E_2$ and accordingly $C_3$ of the
    placebo sample $C_2$ completes the posttest.
  }
  \label{figure:fig.experiment.setup}
\end{figure}

\subsection{Subjects}

The subjects were the 789 latest users of \urort{} which had
signed in with their user name and password on the \urort{} page as of when
the experiment was started.

We wanted about 200 participants to answer our pretest and guessed that
100 would be bothered to install our prototype. Trough randomization
that would ammount to 50 users in the experiment group and 50 users
in the control group.

Since we were only concerned with users of the Firefox web browser we had to
contact a fairly high number of potential respondents to yield a sufficient
number of respondents using this particular browser. We had read that a little
more than 13\% of world wide internet users use Firefox \citep{onestat08}.

\subsection{Data collection}

We collected all our data through questionnaires. An online survey system was
used for creating a pretest, posttest, and follow-up questionnaire. The
questionnaires can be found in their original language and wording in
\appendixref{questionnaire}. What follows are translations to English
of the most important questions and the response options.

\subsubsection{Pretest questions}

These questions were asked only in the pretest. Here we asked questions to
get an impression of our pretest respondents:

\begin{items}
  \item Age{}?\dash{}a numerical value was expected.
  \item Gender{}?\dash{}either male or female.
  \item Firefox user{}?\dash{}selection between the following frequency of use
    categories: always, regularly, sometimes, and seldom/never.
  \item How often do you use \urort{}?\dash{}selection between the following
    frequency categories: daily, several times a week, weekly, monthly,
    and seldom/never.
  \item Do you sign-in (with user name and password) when using
    \urort{}?\dash{}selection between the following frequency
    categories: always, regularly, sometimes, and seldom/never.
\end{items}

We did also ask an open-ended question to investigate if the respondents had
any ideas about new features for \urort{} which would make it easier to
be up-to-date on the latest developments of ones favorites:

\begin{items}
  \item Do you have any wishes for how \urort{} could make it easier to
    keep up-to-date on favorites?
\end{items}

\subsubsection{Posttest questions}

These questions were asked only in the posttest. First we asked specifically
about usage of the prototype implementation.

\begin{items}
  \item How frequently have you used \latest{} when you are
    signed-in on \urort{}?\dash{}selection between the following
    frequency of use categories: have not used, only a few times, almost
    every time, and every time.
\end{items}

Then we asked an open-ended question to investigate how \latest{} influenced
usage of \urort{}:

\begin{items}
  \item How does \latest{} influence your usage of \urort{}?
\end{items}

Next we wanted to investigate the perceived usefulness and ease of use for
our prototype implementation. This well tested approach for conveying
technological acceptance was introduced by \citet{davis89}.
Like \citet[\p{340}]{davis89} we used a 7-point scale as possible answers:

\begin{items}
  \item Extremely unlikely
  \item Unlikely
  \item Slight unlikely
  \item Neutral
  \item Slight likely
  \item Likely
  \item Extremely Likely
\end{items}

We shorted the statements of perceived usefulness down to four alternatives
which we felt made sense for our implementation:

\begin{items}
  \item \latest{} would enable me to keep up-to-date on my favorites in an
    efficient manner.
  \item \latest{} would enable me to keep up-to-date on more favorites.
  \item \latest{} would make it easier to keep up-to-date on favorites.
  \item \latest{} would be useful for keeping up-to-date on favorites.
\end{items}

The statement for perceived ease of use was also shorted down to four
alternatives:

\begin{items}
  \item It would be easy to learn to use \latest{}.
  \item It would be easy to make \latest{} do what I want.
  \item It would be easy to become skillful at using \latest{}.
  \item \latest{} would be easy to use.
\end{items}

Lastly we wanted to investigate if respondents would like \latest{} to be a
standard feature on \urort{}.
We used a 5-point Likert scale \citep{likert32} to gauge respondents
reactions:

\begin{items}
  \item Fully disagree
  \item Somewhat disagree
  \item Neither agree nor disagree
  \item Somewhat agree
  \item Fully agree
\end{items}

With the specific question:

\begin{items}
  \item Do you think \latest{} should be a standard feature of \urort{}?
\end{items}

\subsubsection{Pretest \oldand posttest questions}

These questions were asked both in the pretest and posttest. First we asked
questions to investigate the usage of favorites on \urort{} by our
respondents:

\begin{items}
  \item How many favorites do you have on \urort{}?\dash{}a numerical value
    of the amount of favorites the respondent had was expected.
  \item What makes you add artist on \urort{} as favorites?\dash{}the
    respondent could select one or more of preset reasons or provide their
    own. The preset reasons were: the artist's \emph{music},
    the artist's \emph{popularity}, \emph{friendship} with the artist,
    and \emph{knowledge} of the artist.
  \item How often do you update yourself on what your favorites on \urort{}
    are doing?\dash{}the respondents could select between the following
    frequency categories: daily, several times a week, weekly, monthly,
    and seldom/never.
\end{items}

Then we asked the respondents to qualify several statements which investigated
how easy it was for them to keep up-to-date on favorites.
These statements
was to be ranked on a 5-point Likert scale \citep{likert32}:

\begin{items}
  \item Fully disagree
  \item Somewhat disagree
  \item Neither agree nor disagree
  \item Somewhat agree
  \item Fully agree
\end{items}

And the statements:

\begin{items}
  \item It's easy to keep up-to-date on what my favorites are doing
    on \urort{}.
  \item It's easy to keep up-to-date on whether my favorites publishes
    new songs on \urort{}.
  \item It's easy to keep up-to-date on whether my favorites publishes
    new blog posts on \urort{}.
  \item It's easy to keep up-to-date on whether my favorites are
    performing at concerts.
  \item It's easy to keep up-to-date on the reactions other users at
    \urort{} have towards my favorite artists' songs.
\end{items}

\subsubsection{Follow-up questions}

We asked a small set of questions of how the installation process went. This
was done to investigate how easy or hard installation of Greasemonkey based
prototypes were:

\begin{items}
  \item Did you manage to install \latest{}?\dash{}the respondent had to
  choose between these categories: yes\dash{}it was an easy and quick process,
  yes\dash{}but I experienced small problems, yes\dash{}but I experienced
  large problems, and no\dash{}I gave up.
\end{items}

An open ended question was asked to get more detail in case respondents had
problems installing the Greasemonkey based prototype:


\begin{items}
  \item Please explain your problems with installing \latest{}, if you
    experienced any.
\end{items}

\subsection{Data analysis}

We used two statistical tests for analysing the data we collected. Due to the
limitations a master thesis inherits we did not analyse the shapes of the
distribution for our data. We have therefore not made any assumptions about
the probability distribution of our data. This makes
\term{non-parametric} statistical tests a natural fit as they don't make any
assumptions about the distribution of the variables to be tested
\citep{wikipedia08nonparametric}.

Most of our data are of an ordianal measurement\dash{}meaning that one can
infer the ranking of variables, but not the distance between rankings.
To represent the central tendency of our data we'll therefore use the median
as the mean makes little sence for ordianal data
\citep{wikipedia08levelofmeasurement}.
Ordinal data is another characteristic where non-parametric tests
are best suited \citep{wikipedia08nonparametric}. We therefore decided
to use the two following non-parametric statistical test.

\subsubsection{Mann-Whitney U-test}

The Mann-Whiteney U-test is a non-parametric test for comparing two
independent conditions and can be seen as the non-parametric alternative to
the famous independent t-test \citep[\p{522}]{field05}.
According to \citet[\chap{11a}]{lowry08} the assumptions of the
Mann-Whitney U-test are:

\begin{enum}
  \item The two samples under test should be randomly and independently drawn.
  \item The dependent variable should be intrinsically continuous.
  \item The measures of the two samples should be atleast of an ordinal scale.
\end{enum}

\subsubsection{Wilcoxon signed-rank test}

The Wilcoxon signed-rank test is a non-parametric test for comparing two
related conditions and is an non-parametric alternative to the
dependent t-test \citep[\p{534}]{field05}.
The assumptions of the Wilcoxon signed-rank test are similar to those
of the Mann-Whitney U-test \citep[\chap{12a}]{lowry08}:

\begin{enum}
  \item The paried values of $X_A$ and $X_B$
    should be randomly and independently drawn.
  \item The dependent variable should be intrinsically continuous.
  \item The measures of the paried values $X_A$ and $X_B$
    should be atleast of an ordinal scale.
\end{enum}

%%
%% p < 0.05 considered significant
%%

\subsubsection{Outliers}

We went trough the collected data looking for outliers
\dash{}\postquote{rowntree00}{%
  extreme (high or low) values}
We were only concerned with such numerical outliers for ratio values
(age, number of favorites) where it would be meaningful to talk about means.
For ordinal values we were not concerned with such outliers as we operated
with medians for conveying the center of a distribution.
If outliers were found, the value was simply deleted from the sample.

Another form of outliers can be respondents which respond very
monotonous\dash{}indicating that they just tick off questions without any
thought. In such cases the respondents motivation is to complete the
questionnaire as fast as possible. We did not eliminate such responses from
the sample as it's hard to know the exact motivations of the respondents.
In addition we don't have a particular large set of values for each respondent
to base such analysis on and it would be hard to distinguish such outliers.
