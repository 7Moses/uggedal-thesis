\chapter{Introduction}

% The point of the introduction is to answer: what is this thesis about?
% Explain this in four steps by:
%
%   * Why you have chosen this topic rather than any other. Examples:
%       - has been neglected
%       - much discussed but not properly and fully
%   * Why this topic interests you.
%   * The kind of research approach or academic disciple you will utilize.
%   * Your research questions or problems.
%
% The role of the introduction, like the abstract, is to orient your readers.
% This is best done clearly and succinctly.

The web has come a long way since it's inception when it functioned as a
global interconnected system for document sharing amongst researchers
\citep[p.~82]{bernerslee92}. We've
seen the coming of an increasingly more social web as
``the digital domain has seen a significant growth in the scale and richness
of on-line communities'' \citep[p.~44]{backstrom06}.
\citet[ch.~1, p.~2]{rosa07} reports that the
use of blogs in the general public%
\sidenote{%
  Represented with a total of 6,545 respondents
  to a survey conducted in Canada, France, Germany, Japan,
  the United Kingdom, and the United States.
}
have grown significantly between two surveys conducted in 2005 and 2007,
roughly 18 months apart.
More than one third of people using blogs surveyed in 2007 contributes
actively by writing blog articles or commenting on blog posts as can be seen in
\tablepageref{blog.usage}
\citep[ch.~1, p.~6]{rosa07}.
It has been argued that web citizens'
familiarity with blogging laid the groundwork for the explosion we are seeing
in user participation in web communities \citep{weiss05,beer07}.

At the same time advances in hardware and web development tools have made it
easier and cheaper to create new web sites. We're now seeing an
abundance of new offerings in this field.
During the initial studies of our research we frequented
many of the these web sites. Our impression is that
this area of the web infamously coined \emph{Web 2.0}%
\sidenote[-4\baselineskip]{%
  Web 2.0 was first used as the name of a conference arranged by
  O'Reilly Media. The ``2.0'' part of the conference name was then used to
  signify the revival of interest in the web after the dot-com bubble in the
  early 21st century \citep{oreilly07}.
  Later the founder of O'Reilly Media, Tim O'Reilly, defined
  the term as the characteristics of the web sites that survived the dot-com
  bubble and the web sites he deemed to be the best newcomers to the
  field \citep{oreilly05}.
}
is bringing interesting innovations. As \citet[p.~18]{weiss05} argues:
``When we consider a hot, buzz-worthy Web site of the new Internet evolution
\ldots
they are at the same time incredibly innovative and yet--not''.
User generated content and the notion of \emph{collective intelligence} is
driving the innovation we're seeing in this incrementally new version of the
web. But basically what we're experiencing today with social software and
collaborative systems was envisioned by \citet{bush45} and
\citet{licklider68} years before the Web and even the Internet.

\section{Focus}

This thesis have a focus on navigational problems and only those
which are of a social nature. To see what we mean when we talk about
navigational systems which supports sociality take a look at
\sectionref{background.social.navigation}.
Navigation in context of computer systems is
essentially a metaphor based on how people find their way in the physical
world. So just as a compass and map can be crucial in your ability to find a
cabin deep in the woods during a hike--reliable and efficient navigational
systems on the web is of uttermost importance when you're trying to locate a
certain electronic object containing valuable information.

\sidetable{Blog Usage}{%
  \label{table:blog.usage}
  \begin{tabular}{llp{1.2cm}}

    \toprule
    Year & Total & Active \\
         &       & of total \\
    \midrule

    2005 & 16\% & Not \\
         &      & known \\
    2007 & 45\% & 17\% \\

  \end{tabular}
}

In addition to only focusing on \emph{social navigation} we're only concerned
which such types navigation on the Web.
On the Web we're using hyperlinks \citep[p.~90]{nelson65} to provide users
with navigational choices. We're only focusing on the use of such hyperlinks
within web browsers and not navigation support in auxiliary tools as email
clients, instant messaging clients, and so on. While focusing on the browser,
we target our research only on what happens inside various web pages. This
means that other navigation forms supported by the browser itself or third
party extensions or plugins is outside of our scope.
% this is also discussed in the background chapter with a citation.
% maby it's ok to introduce it here and restate it in the background chapter
% where citations are used.

When studying various web pages it bacame apparent that some use of
social navitaion mechanisms implies pretty large privacy concerns. By mining
users' previous actions specific user profiles can be generated. One can then
represent very sencitive characteristics of individuals such as sexual
orientation, political status, religious beliefs, and so on.
We feel this subject area of social navition in relation to privacy warrants a
master thesis on it's own. Discussion of privacy concerns have therefore been
excluded from our research.

\section{Motivation}

Social navigation are as we've seen a well defined term within academia.
During our literature review we collected to the best of our abilities all
academic articles where social navigation was discussed. Our approach was to
use keyword search and citation search in the databases listed in
\tablepageref{literature.databases}
\tableref{social.navigation.academia}
shows the metrics of articles
we found about social navigation in context of the modern web as captured by
the Web 2.0 term (social networking sites, folksonomies, and wikis) and other
areas of computer science (classic web, general user interfaces, security, and
so on).
On several occasions we encountered similar articles by the same authors
discussing the same problems and systems. In such circumstances the collection
of two or more articles was counted as one.

\sidetable{Social Navigation in Academia}{%
  \label{table:social.navigation.academia}
  \begin{tabular}{ll}

    \toprule
    Context & Articles \\
    \midrule

    Modern Web & 5 \\
    Other & 21 \\

  \end{tabular}
}

Our current area of Web 2.0 in relation to navigational problems have in our
view small coverage in academia.
As \citet{beer07} notes: ``\ldots `internet time' now runs at at a clock speed
several orders of magnitude faster than that of academic research''.
We described earlier the growth we're seeing of web sites with social
aspects and we believe that some of these provide for novel examples of social
navigation. It would therefore be interesting to look at some of the
state-of-the-art social web sites and look at what contributions they have
made to the field of social navigation.

We are currently lacking information on how one can use social navigation
consciously in a modern web application design process. Such navigational
schemes seem to be created without guidance and many times as an afterthought.
It appears that methods for establishing incentives for user participation
is the focal point of web architecture design today. Even though such
approaches can result in sound and interesting navigation it's our impression
that a focus on solving users' navigational problems is more beneficial for
the usability of a web sites.

\section{Objective}

By collecting examples of good navigational implementations in the wild,
analyzing them, and categorize them we hope to offer a structured view of the
field of social navigation. We offer this information in the way of a taxonomy
of useful social navigation techniques.
As we are unaware of any established technique for
conducting such a study on real world navigation systems we create our own
method as we go---fine tuning it as we learn from our experiences.

We try to improve an existing web site by using some of the techniques
established in our initial research of how social navigation is leveraged
in real world applications. More specifically we are prototyping possible
navigational improvements for the Norwegian Broadcasting Corporation's joint
TV, radio, and Internet project: \emph{Ur\o{}rt}---a site where artists upload
their demos and get valuable playtime on radio and TV if their products are
judged to be of sufficient quality. Our focus is on the projects
web community%
\sidenote{
  Available at \url{http://nrk.no/urort}
}
where users can interact in a social manner in addition to uploading
their songs.

Going in and making changes to an existing web site can be both an
daunting and time consuming task. First one have to establish a trustworthy
relationship with the creators of such a site so that they are certain
you're not introducing bugs in their production software. Secondly, grasping the
code base, third party libraries, and development tools of such a software
project demands a lot of upfront effort before any real development work can
begin. This goes against the prototypical process we intended to use while
experimenting with Ur\o{}rt.

Even though we've had an ongoing dialog with the developers of Ur\o{}rt we
decided to create our prototype as a layer on top of their site. By using an
extension%
\sidenote{
  Specifically \emph{Greasemonkey}, an extension allowing for
  customization of presentation and behavior of web pages. Available at
  \url{http://www.greasespot.net}
}
for the open source Firefox%
\sidenote{
  Available at \url{http://firefox.com}
}
web browser we were able to make changes and additions to the way Ur\o{}rt were
presented to users who were participating in our study. We were able to create
a back end for the additional data and behavior our new functionality required
with the frameworks and programming language we found to be most efficient in
a prototypical process. This resulted in a transparent experience for our end
users as long as they had taken the necessary steps to set up the browser
extension and our script.

With our technical solution in place we were able to test how it performed in
practice by conducting ``some sort of user observation'' (not decided) and
(possibly) more quantitative surveys.

This leads to the research question we had in mind while conducting
the tasks described above:

\begin{quote}
  How can social navigation influence usage of established web sites?
\end{quote}

The role of this question have been to give our research direction, show where
it's boundaries were, keep us focused, and point to the needed methods and
data \citep[p.~77]{silverman05}.

% hypotheses ?

\section{Contributions}

Contributions from our research on social navigation is threefold:

\begin{description}
  \item[Informing navigational design] by giving a structured overview of
    various social navigational schemes in use today.
  \item[Transparent prototyping methods] by sharing experiences with
    creating an unobtrusive shell of navigational designs on top of an
    existing web site.
  \item[Applicability of social navigation] by discussing results from
    a study of real world usage of social navigation.
\end{description}

\section{Outline}

Moving on from this introductory chapter we'll introduce the background
material needed for understanding the field of social
navigation in \chapterref{background}, look at what methods we've
used for data collection and subsequent data analysis in
\chapterref{methodology}, the actual survey and analysis of the data
we've collected in \chapterref{analysis}, a discussion of the
broader lines of our analysis is presented in \chapterref{discussion}
before we finally reflect on our research and possibilities for future work
in \chapterref{conclusion}.
