\chapter{Source Code}

\section{Reddit Collaborative Filtering Algorithm}
\label{section:source.code.reddit}

Here is the source code of the algorithm that decides the score of a
submission. We made syntactical changes to make the code's intent clearer.

\begin{scode}{Python}{reddit.cf.algorithm}{%
  Reddit Collaborative Filtering Algorithm}{%
  The collaborative filtering algorithm used on Reddit}
\begin{lstlisting}
from datetime import datetime, timedelta
from math import log

def seconds_since_cutoff(date):
    cutoff = datetime(2005, 12, 8, 7, 46, 43)
    td = date - cutoff

    seconds = td.days * 24 * 60 * 60
    seconds += td.seconds
    seconds += float(td.microseconds) / 1000000
    return seconds

def score(up_votes, down_votes, submitted_date):
    score = up_votes - down_votes
    order = log(max(abs(score), 1), 10)
    sign = 1 if score > 0 else -1 if score < 0 else 0

    age = seconds_since_cutoff(submitted_date)

    return round(order + sign * age / 45000, 7)
\end{lstlisting}
\end{scode}


\section{JavaScript Comment Stripper}
\label{section:source.code.javascript.comment.stripper}

This small script was written to help
compare different JavaScript libraries:


\begin{scode}{Ruby}{javascript.comment.sripping}{%
  JavaScript Comment Stripping}{%
  Strips comments from JavaScript libraries}
\begin{lstlisting}
#!/usr/bin/env ruby

js_files = Dir['*.js']

ignore_pattern = /^[\s\t]?(\/\*|\*|\/\/)/

js_files.each do |file|
  File.open("#{file}.out", 'w') do |out| 
    out.puts File.readlines(file).reject do |line|
      line =~ pattern
    end
  end
end
\end{lstlisting}
\end{scode}

\section{Shell File and Directory Hierarchy}
\label{section:source.code.hierarchy}

This one-liner was used for conveying the directory and file hierarchy of our
server side software:

\begin{scode}{sh}{file.directory.hierarchy}{%
  \abbr{UNIX} File and Directory Hierarchy}{%
  File and directory hierarchy with standard \abbr{UNIX} tools}
\begin{lstlisting}
find . | sed -e 's/[^\/]*\//|--/g' -e 's/-- |/|/g'
\end{lstlisting}
\end{scode}


