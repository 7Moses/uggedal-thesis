\clearpage
\clearpage

\section{Discussion}

\subsection{Activity streams as a social navigation technique}

\subsection{Transparent prototyping with Greasemonkey}


When we conducted a study of our prototype with real world users
we got valuable feedback on how well such a system works for the average user.

\subsubsection{Limited in browser selection}

% Reference how many we had to send mail to and how many answered.
% Reference firefox usage for respondents to remove those respondents
% which did not actively use firefox from the above equation.

\subsubsection{Difficulties with installing Greasemonkey and user-scripts}

% reference non accomplish rates

\parabreak

During our development of a prototype application with Greasemonkey for
enhancing an established web page we got a feel for its pros and cons from a
development perspective.

\subsubsection{Requires no access to the established implementation}

\subsubsection{Requires little knowledge of the established implementation}

\subsubsection{Requires more work than altering the established
  implementation}

\subsubsection{Fragile when the established implementation is changed}
% Only happened once during a two month span on \urort{}. What was scary
% thought was that the changes made the user script on the client side
% obsolete. If this had happened under production usage when the user scripts
% was pushed to the clients we would be in a world of trouble. Changes on the
% server side platform can be handled more transparently.

\subsubsection{Less performant than the established implementation}

\section{Generalizability and Validity}

% We were only concerned with Firefox users. These are probably
% not representative for Firefox users.
%
% The installation process was somewhat complicated. This means
% that the most technically knowledgeable people are those
% which answered our posttests.
%
% The people who want to participate in such an elaborate study is
% probably fairly interested in Urørt. They are experienced Urørt users
% and are therefore not representative of the Urørt population.
%
% This has consequences for the validity of our study. We can not generalize
% back to Urørt users in general.
