\chapter{Discussion}
\label{chapter:discussion}

This chapter will include a synthesis based on the analysis of our collected
data. The larger lines of our findings will be presented and we'll try to
represent a new and refreshed terminology for the field of social navigation
by the means of a taxonomy including patterns of navigational use.

\section{Social Navigation on the Social Web}

\subsection{No Explicit Design for Social Navigation}

It does not seem like the designers of social web pages design for social
navigation explicitly. There seem to be a trend for designing for social
interaction. An implicit by-product of such an approach seems to be
the creation of several forms of social navigation constructs.
We base this observation our studies of two large social web sites in
\chapterref{analysis} in addition to cursory observation on other social web
sites.

\subsection{Friendship, Group, and Content-Based Social Navigation}

% Possible themes could be a discussion of grouping of all social navigation
% as either friendship and group based or content based. A hypothesis needs to
% be established and we'll have to try falsify this claim based on our
% knowledge of the field after conducting both throughout analysis and more
% cursory analysis of several web pages.

It seems to us that further distinctions from those presented in
\sectionref{social.navigation.fundamental.categorization}
of different kinds of social navigation is in order. We have seen that social
navigation can either be based on:

\begin{items}
  \iterm{friends} you have a connection with,
  \iterm{groups} you belong to, or
  \iterm{content} that you seem to seek.
\end{items}

\subsection{Social Navigation is Peer-Based}

An overlying theme of the forms of social navigation we found in the wild were
that it was created by equal peers. The construct for enabeling the social
navigation are created by the web site creators. But the data that
enables navigational choices of a social nature are created by other users
like you\dash{}by the community.

\section{Transparent Prototyping with Greasemonkey}

\subsection{For Developers}

During our development of a prototype application with Greasemonkey for
enhancing an established web page we got a feel for it's pros and cons from a
development perspective.

\subsubsection{Requires No Access to the Esablished Implementation}

\subsubsection{Requires Little Knowledge of the Established Implementation}

\subsubsection{Requires More Work than Altering the Established Implementation}

\subsubsection{Fragile when the Established Implementation is Changed}


\subsection{For Users}

When we conducted a study of our prototype application with real world users
we got valuable feedback on how well such a system works for the average user.
