\chapter{Discussion}
\label{chapter:discussion}

This chapter will start with a synthesis based on the analysis of our
collected data on social navigation in modern web sites.
Next we'll discuss how our technique for transparently prototyping
applications relates to both developers and users. Lastly we'll look at how
activity feeds fares as a social navigation technique.

\section{Social Navigation on the Social Web}

\subsection{No explicit design for social navigation}

It does not seem like the designers of social web pages design for social
navigation explicitly. There seem to be a trend for designing for social
interaction. An implicit by-product of such an approach seems to be
the creation of several forms of social navigation constructs.
We base this observation our studies of two large social web sites in
\chapterref{analysis} in addition to cursory observation on other social web
sites.

\subsection{Social navigation advice is given by peers}

An overlying theme of the forms of social navigation we found in the wild were
that it was created by equal peers. The construct for enabling the social
navigation are created by the web site creators. But the data that
enables navigational choices of a social nature are created by other users
like you\dash{}by the community.

Whether indirect or direct, explicit or implicit, navigational advice have to
be given by peers to be considered social navigation. In other words
navigational advice have to be given by individuals on the same horizontal
level as yourself. This means that navigational advice given by web editors
and web designers\dash{}people vertically superior to yourself\dash{}can not
be true social navigation.

% From background about GGG:
% In other terms this means that social relationships on the Web have become
% so important that they're more interesting themselves than the pages that
% represents them. While it would be very interesting to look at how social
%
% Tie this in here about social relationships importance and social
% connections related to peers.

%%
%% elaborate
%%

%%
%% persistant structures found in the real world can afford social navigation
%% in our virtual world. -- Robins
%%

\section{Transparent Prototyping with Greasemonkey}

\subsection{Development perspective}

During our development of a prototype application with Greasemonkey for
enhancing an established web page we got a feel for its pros and cons from a
development perspective.

\subsubsection{Requires no access to the established implementation}

\subsubsection{Requires little knowledge of the established implementation}

\subsubsection{Requires more work than altering the established
  implementation}

\subsubsection{Fragile when the established implementation is changed}
% Only happened once during a two month span on \urort{}. What was scary
% thought was that the changes made the user script on the client side
% obsolete. If this had happened under production usage when the user scripts
% was pushed to the clients we would be in a world of trouble. Changes on the
% server side platform can be handled more transparently.

\subsubsection{Less performant than the established implementation}

\subsection{User perspective}

When we conducted a study of our prototype with real world users
we got valuable feedback on how well such a system works for the average user.

\begin{table}[h]
  \begin{tabular}{lrr}

    &
    \multicolumn{1}{c}{Individuals} &
    \multicolumn{1}{c}{\% of respondents} \\

    \cmidrule(lr){2-2}
    \cmidrule(lr){3-3}

    Respondents &
    125 &
    \\

    Completed installation &
    72 &
    57.6 \\

    Used application &
    45 &
    36.0 \\

  \end{tabular}
  \caption[Respondents Falloff]{%
    Respondent falloff during installation and use of our prototype
    application}
  \label{table:respondents.falloff}
\end{table}

\subsubsection{Limited in browser selection}

\subsubsection{Difficulties with installing Greasemonkey}

\subsubsection{Difficulties with installing user-scripts}


\section{Activity Feeds as a Social Navigation Technique}
