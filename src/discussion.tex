\chapter{Discussion}
\label{chapter:discussion}

This chapter will include a synthesis based on the analysis of our collected
data. The larger lines of our findings will be presented and we'll try to
represent a new and refreshed terminology for the field of social navigation
by the means of a taxonomy including patterns of navigational use.

\section{Social Navigation on the Social Web}

\subsection{No Explicit Design for Social Navigation}

It does not seem like the designers of social web pages design for social
navigation explicitly. There seem to be a trend for designing for social
interaction. An implicit by-product of such an approach seems to be
the creation of several forms of social navigation constructs.
We base this observation our studies of two large social web sites in
\chapterref{analysis} in addition to cursory observation on other social web
sites.

\subsection{Social Navigation Advice is Given by Peers}

An overlying theme of the forms of social navigation we found in the wild were
that it was created by equal peers. The construct for enabling the social
navigation are created by the web site creators. But the data that
enables navigational choices of a social nature are created by other users
like you\dash{}by the community.

Whether indirect or direct, explicit or implicit, navigational advice have to
be given by peers to be considered social navigation. In other words
navigational advice have to be given by individuals on the same horizontal
level as yourself. This means that navigational advice given by web editors
and web designers\dash{}people vertically superior to yourself\dash{}can not
be true social navigation.

%%
%% elaborate
%%


\section{Transparent Prototyping with Greasemonkey}

\subsection{For Developers}

During our development of a prototype application with Greasemonkey for
enhancing an established web page we got a feel for its pros and cons from a
development perspective.

\subsubsection{Requires No Access to the Established Implementation}

\subsubsection{Requires Little Knowledge of the Established Implementation}

\subsubsection{Requires More Work than Altering the Established Implementation}

\subsubsection{Fragile when the Established Implementation is Changed}
% Only happened once during a two month span on \urort{}. What was scary
% thought was that the changes made the user script on the client side
% obselete. If this had happened under production usage when the user scripts
% was pushed to the clients we would be in a world of trouble. Changes on the
% server side plattform can be handeled more transparently.

\subsubsection{Less Performant than the Established Implementation}

\subsection{For Users}

When we conducted a study of our prototype application with real world users
we got valuable feedback on how well such a system works for the average user.


\subsubsection{Limited in Browser Selection}

\subsubsection{Difficulties with Installing Greasemonkey}

\subsubsection{Difficulties with Installing User-Scripts}
