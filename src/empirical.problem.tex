\section{Research Problem and Hyoptheses}

\begin{quote}
  Can social navigation trough activity streams influence
  usage of an established web site?
\end{quote}

This question deals with how a specific social navigation technique, activity
streams, could potentially influence experiment participants' usage of a web
site.
From this research question we also proposed more specific problem statements
that more clearly states different ways of influencing usage:

\begin{quote}
  Can social navigation trough activity streams help users keep
  up-to-date on favorites' activities on \urort{}?
\end{quote}

This question deals with the way users keep up-to-date on what their favorites
are doing on \urort{}. We want to investigate if activity streams can help
users in this task.

We were also conserned with how activity streams influenced the
frequency of keeping up-to-date on activities:

\begin{quote}
  Does social navigation trough activity streams lead users to more often keep
  up-to-date on favorites' activities on \urort{}?
\end{quote}

The next research question deals with with how activity streams could
potentially influence the importance of favorites on \urort{}:

\begin{quote}
  Does social navigation trough activity streams lead users to make
  more artists on \urort{} their favorites?
\end{quote}

We also had a more technical research question relating to how one can
conduct experiments on established web sites with Greasemonkey:

\begin{quote}
  Can navigational prototyping with Greasemonkey be considered a
  viable technical option when testing user behaviour in an
  established web site?
\end{quote}

This qustion investigates whether Greasemonkey prototyping should be
considered as one of potentially many technical alternatives in future
research experiments where user behaviour is tested. As this is a new way
to test user behaviour in relation to social navigation we're mainly concerned
with gaining experience with using such a technical solution.

\parabreak

From these research problems we created several hypotheses.

\subsection{Keeping up-to-date on favorites' activities}

Our main hypothesis deals with how easy respondents can keep-up-to
date with favorites' activities in general with and without an 
activity stream.
$B$ is the degree respondents can easily keep up-to-date with
favorites' activities without an activity stream and $A$ is the degree
respondents can easily keep-up-to-date with favorites' activites  with
an activity stream.
\begin{items}
  \iterm{$H_0$:} $Mdn_{\sum{B}} \geq Mdn_{\sum{A}}$
  \iterm{$H_A$:} $Mdn_{\sum{B}} < Mdn_{\sum{A}}$
\end{items}

From the main hypothesis on activities we have several more specific
hypotheses that deals with specific activity types.

$B$ is the degree respondents can easily keep up-to-date with
favorites' specific activities
(published songs, blog posts, concert apperances, and song reviews)
without an activity stream and $A$ is the degree
respondents can easily keep-up-to-date with favorites' specific
activities with an activity stream.
\begin{items}
  \iterm{$H_0$:} $Mdn_{\sum{B}} \geq Mdn_{\sum{A}}$
  \iterm{$H_A$:} $Mdn_{\sum{B}} < Mdn_{\sum{A}}$
\end{items}

Relating to the hypotheses about the degree respondents can keep
up-to-date on activities we have a hypothesis concerning
the frequency of keeping up-to-date.

$B$ is the frequency respondents keep up-to-date on favorites' activities
without an activity stream and $A$ is the frequency respondents keep
up-to-date on favorites' activities with an activity stream.
\begin{items}
  \iterm{$H_0$:} $Mdn_{\sum{B}} \geq Mdn_{\sum{A}}$
  \iterm{$H_A$:} $Mdn_{\sum{B}} < Mdn_{\sum{A}}$
\end{items}
