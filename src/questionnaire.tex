\chapter{Questionnaire}
\label{appendix:questionnaire}

Questions:

\begin{items}
  \item Acquiescence bias: response bias in which respondents
    to a survey have a tendency to agree with all the questions or to indicate
    a positive connotation.
  \item Social desirability bias: the inclination to present oneself in a
    manner that will be viewed favorably by others.
  \item Response scale: Likert, numbered (0-10?), more?
  \item Level of measurement: nominal (gender, uniquely name the measurement,
    no order), ordinal (preference, odered by rank, but distance between
    elements are meaningless), interval (distance between elements have a
    meaning, but ratios are meaningless), ratio (age, always an absolute zero
    which is meaningful). Generally nominal measurements can use the median
    (but not the mean) to find the central tendency.
    Ok with mixed levels?
\end{items}

\section{User Profile}

Questions to determine who the respondent is and his or hers usage of
\urort{}.

\begin{items}
  \item Hvor gammel er du? [0-19, 20-29, 30-40, 40-]
  \item Hvilket kjønn er du? [mann, kvinne]
  \item Hvor ofte bruker du \urort{}?
    [daglig, ukentlig, månedlig, skjelden, aldri]
  \item I hvilken tilstand benytter du oftest \urort{}? [innlogget, anonymt]
  \item Hvor ofte bruker du tjenester på web som støtter sosiale
    nettverk (eksempelvis Facebook, Myspace, last.fm, Flickr, Underskog
    og så videre)?
    [daglig, uketlig, månedlig, skjelden, aldri]
\end{items}

\section{Favorites on \urort{}}

Questions to determine the respondent's familiarity with favorites and use of
favorites on \urort{}.

\begin{items}
  \item Er du kjent med begrepet \q{favoritter} på \urort{}? [ja, nei]
  \item Hvor mange favoritter har du på \urort{}? [0, 1-9, 10-20, 20-50, 50-]
  \item Hva slags kriterier benytter du når du velger favoritter på
    \urort{}? [musikk, kjennskap, vennskap, popularitet]
  \item Hvor ofte bruker du favoritter på \urort{}?
    [daglig, ukentlig, månedlig, skjelden, aldri]
  \item Hva bruker du favoritter til på \urort{}?
    [holde meg oppdatert, vise støtte, annet]
\end{items}

\section{Social Navigation Provided by Favorites}

Questions to determine the respondent's use of
social navigation provided by favorites on \urort{}.

\begin{items}
  \item Nøytral:
  \item Hvordan synes du det er å holde seg oppdatert med hva dine favoritter
    foretar seg på \urort{}?
    [veldig enkelt, ganske enkelt, litt enkelt,
    verken enkelt eller vanskelig,
    litt vankselig, ganske vanskelig, veldig vanskelig]
  \item Positiv:
  \item Jeg synes det er enkelt å finne ut hva mine favoritter
    foretar seg på \urort{}.
    [veldig enig, ganske enig, litt enig,
    verken enig eller uenig,
    litt uenig, ganske uenig, veldig uenig]
  \item Negativ:
  \item Jeg synes det er vanskelig å holde meg oppdatert på mine
    favoritters handlinger på \urort{}.
    [veldig enig, ganske enig, litt enig,
    verken enig eller uenig,
    litt uenig, ganske uenig, veldig uenig]
  \item Jeg har god oversikt over de nyeste sangene mine favoritter
    har lagt ut på \urort{}.
    [veldig enig, ganske enig, litt enig,
    verken enig eller uenig,
    litt uenig, ganske uenig, veldig uenig]
  \item Type spesifik:
  \item Jeg har god oversikt over hvordan andre \urort{} brukere reagerer på
    sangene mine favoritter har gjort tilgjengelige.
    [veldig enig, ganske enig, litt enig,
    verken enig eller uenig,
    litt uenig, ganske uenig, veldig uenig]
  \item Jeg har god oversikt over de seneste blogg-innleggene
    mine favoritter har lagt ut på \urort{}.
    [veldig enig, ganske enig, litt enig,
    verken enig eller uenig,
    litt uenig, ganske uenig, veldig uenig]
  \item Jeg har god oversikt over hvilke konserter mine favoritter
    på \urort{} holder.
    [veldig enig, ganske enig, litt enig,
    verken enig eller uenig,
    litt uenig, ganske uenig, veldig uenig]
\end{items}

\section{Usefulness}

Questions to determine the respondent's percieved usefulness of new
functionality before and after actual usage.

\begin{items}
  \item Bruk av \siste{} vil la meg fullføre mine oppgaver på \urort{}
    raskere.
    [veldig sannsynlig, ganske sannsynlig, litt sannsynlig,
    verken sannsynlig eller usannsynlig,
    litt usannsynlig, ganske usannsynlig, veldig usannsynlig]
  \item Bruk av \siste{} vil øke min produktivitet på \urort{}.
    [veldig sannsynlig, ganske sannsynlig, litt sannsynlig,
    verken sannsynlig eller usannsynlig,
    litt usannsynlig, ganske usannsynlig, veldig usannsynlig]
  \item Bruk av \siste{} vil gjøre bruken av \urort{} lettere.
    [veldig sannsynlig, ganske sannsynlig, litt sannsynlig,
    verken sannsynlig eller usannsynlig,
    litt usannsynlig, ganske usannsynlig, veldig usannsynlig]
  \item Jeg ville finne \siste{} nyttig på \urort{}.
    [veldig sannsynlig, ganske sannsynlig, litt sannsynlig,
    verken sannsynlig eller usannsynlig,
    litt usannsynlig, ganske usannsynlig, veldig usannsynlig]
\end{items}

\section{Ease of Use}

Questions to determine the respondent's percieved ease of use for new
functionality before and after actual usage.

\begin{items}
  \item Lære å bruke \siste{} ville være lett for meg på \urort{}.
    [veldig sannsynlig, ganske sannsynlig, litt sannsynlig,
    verken sannsynlig eller usannsynlig,
    litt usannsynlig, ganske usannsynlig, veldig usannsynlig]
  \item Min interaksjon med \siste{} ville være klar og forståelig
    på \urort{}.
    [veldig sannsynlig, ganske sannsynlig, litt sannsynlig,
    verken sannsynlig eller usannsynlig,
    litt usannsynlig, ganske usannsynlig, veldig usannsynlig]
  \item Det ville være lett for meg å mestre bruken av \siste{}
    på \urort{}.
    [veldig sannsynlig, ganske sannsynlig, litt sannsynlig,
    verken sannsynlig eller usannsynlig,
    litt usannsynlig, ganske usannsynlig, veldig usannsynlig]
  \item Jeg ville finne \siste{} lett å bruke på \urort{}.
    [veldig sannsynlig, ganske sannsynlig, litt sannsynlig,
    verken sannsynlig eller usannsynlig,
    litt usannsynlig, ganske usannsynlig, veldig usannsynlig]
\end{items}

